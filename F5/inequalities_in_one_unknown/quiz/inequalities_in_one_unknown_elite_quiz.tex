\documentclass[addpoints, 10pt]{exam}
\usepackage[utf8]{inputenc}
\usepackage[T1]{fontenc}
\usepackage{amsmath}
\usepackage{amssymb}
\usepackage{systeme}
\usepackage{svg}
\usepackage{pgfplots}
\usepackage{bm}
\pgfplotsset{width=10cm,compat=1.9}
\title{MATH5.1EL Quiz 2\\ Inequalities in One Unknown \\ Time Limit: (30 minutes)}
\author{T Yeung}
\date{}

\renewcommand{\solutiontitle}{\noindent\textbf{\underline{Solution:}}\par\noindent}

\pagestyle{headandfoot}
\firstpageheader{\normalsize \bfseries Math 5.1EL \\ Quiz 2 \\ 2023-24}{}{}
\firstpageheadrule
\runningheadrule
\runningheader{Math 5.1EL}{Quiz 2}{14 Oct 2023}
\footer{}{Page \thepage\ of \numpages}{}

\begin{document}

\maketitle

\begin{center}
	\fbox{\fbox{\parbox{5.5in}{\centering
				Answer the questions in the spaces provided on the
				question sheets. If you do not know how to answer 
				a certain question, write down where you get stuck.
				Answers can be corrected to 3 significant figures
				if necessary.
	}}}
\end{center}
\vspace{0.1in}
\makebox[\textwidth]{Name, class, class no.:\enspace\hrulefill}
\vspace{0.3in}
\makebox[\textwidth]{Tutor’s name:\enspace\hrulefill}

\begin{questions}
	\marksnotpoints
	\question[4] If the quadratic curve $y=x^2+kx+8$ intersects the straight line $y=4x-1$ at two distinct points, find the range of possible values of $k$.
	\begin{solutionorlines}[9cm]
		\begin{equation*}
			\begin{cases}
				y=x^2+kx+8 & (1) \\
				y=4x-1 & (2)
			\end{cases}
		\end{equation*}
		Substitute (2) into (1),
		\begin{align*}
			4x-1 &= x^2+kx+8 \\
			x^2-(4-k)x+9 &= 0
		\end{align*}
		$\because$ There are two intersections
		\begin{align*}
			\Delta &> 0 \\
			(4-k)^2-4(9) &> 0 \\
			k^2-8k-20 &> 0 \\
			(k-10)(k+2) &> 0 \\
			-2 &< k < 10
		\end{align*}
		$\therefore -2 < k < 10$
	\end{solutionorlines}

	\newpage

	\question[4] If $x^2+k+kx=3$ is always positive for all real values of $k$, find the range of possible values of $k$.
	\begin{solutionorlines}[9cm]
		$\because x^2+kx+(k-3) > 0$ \\
		$\therefore \Delta < 0$
		\begin{align*}
			\Delta &< 0 \\
			k^2-4k+12 &< 0 \\
			(k-6)(k+2) &< 0 \\
			-2 &< k < 6
		\end{align*}
		$\therefore -2 < k < 6$
	\end{solutionorlines}

	\question[3] Solve $(2x-3)(3x+1) \geq 4x(2x-3)$
	\begin{solutionorlines}[9cm]
		\begin{align*}
			(2x-3)(3x+1) \geq 4x(2x-3) \\
			6x^2-7x-3 \geq 8x^2-12x \\
			2x^2-5x+3 \geq 0 \\
			(2x+1)(x-3) \geq 0 \\
			x \leq -\tfrac{1}{2} \text{ or } x \geq 3
		\end{align*}
	\end{solutionorlines}
	
	\newpage

	\question[10] $\alpha$ and $\beta$ are the roots of the quadratic equation $x^2+(p+1)x+(p-1)=0$, where $p$ is real.
	\begin{parts}
		\part[3] Show that $\alpha$ and $\beta$ are real and distinct.
		\part[3] Show that $(\alpha-2)(\beta-2)=3p+5$.
		\part[4] Given that $\beta < 2 < \alpha$,
			\begin{subparts}
				\subpart Using the result of (b), show that $p < -\frac{5}{3}$.
				\subpart If $(\alpha-\beta)^2<24$, find the range of possible values of $p$. Hence write down all possible integral value(s) of $p$.
			\end{subparts}
	\end{parts}
	\begin{solutionorlines}[16cm]
		(a) 
		\begin{align*}
			\Delta &= (p+1)^2-4(p-1)\\
				   &= p^2+2p+1-4p+4  \\
				   &= p^2-2p+5 = (p-1)^2+4 (> 0)
		\end{align*}
		$\therefore$ There are two distinct real solutions.

		(b) By sum of roots and product of roots formula, we have
		\begin{equation*}
			\begin{cases}
				\alpha \beta = p - 1 \\
				\alpha + \beta = -(p + 1) 
			\end{cases}
		\end{equation*}
		\begin{align*}
			(\alpha-2)(\beta-2) &= \alpha \beta - 2(\alpha + \beta) + 4 \\
								&= p - 1 - 2[-(p+1)] + 4 \\
								&= p - 1 + 2p - 2 + 4 \\
								&= 3p + 5 
		\end{align*}
		(c) i. 
			\begin{align*}
				\beta < 2 < \alpha \\
				(\alpha-2)(\beta-2) < 0 \\
				3p+5 < 0 \\
				p < -\tfrac{5}{3}
			\end{align*}
			ii. 
			\begin{align*}
				(\alpha-\beta)^2 &< 24 \\
				(\alpha+\beta)^2-4\alpha\beta &< 24 \\
				[-(p+1)]^2-4(p-1) &< 24 \\
				p^2-2p-19 &< 0 \\
				1-2\sqrt{10} < p < 1 + 2\sqrt{10}  \\
				-5.32 < p < 7.32
			\end{align*}
			 $\because -5.32 < p < -1.66$ \\
			 $\therefore p$ can be $-5,-4,-3,-2$
	\end{solutionorlines}

	\ifprintanswers
	\else
		\newpage
	\fi

	\question[6] Given $x^2-2(1+a)x+(3a^2+4ab+4b^2+2)=0$, where $a$ and $b$ are real.
	\begin{parts}
		\part[3] Show that the discriminant of the equation is $-4[(a-1)^2+(a-2b)^2]$
		\part[3] Find $a$ and $b$ if the equation has equal real roots.
	\end{parts}
	\begin{solutionorlines}[20cm]
		(a) 
		\begin{align*}
			\Delta &= [-2(1+a)]^2x-4(3a^2+4ab+4b^2+2) \\
				   &= 4+8a+4a^2-12a^2-16ab-16b^2-8 \\
				   &= -8a^2-16ab-16b^2+8a-4 \\
				   &= -4[a^2-2a+1+a^2-4ab+4b^2] \\
				   &= -4[(a-1)^2+(a-2b)^2]
		\end{align*}
		(b) 
		\begin{align*}
			\Delta &= 0 \\
			-4[(a-1)^2+(a-2b)^2] &= 0 \\
			a = 1 \text{ and } a = 2b \\
			a = 1 \text{ and } b = \tfrac{1}{2}
		\end{align*}
		$\therefore a = 1 \text{ and } b = \dfrac{1}{2}$
	\end{solutionorlines}
\end{questions}

\end{document}


