\documentclass{beamer}
\usepackage[utf8]{inputenc}
\usepackage{xcolor}
\usepackage{tikz}
\usepackage{tkz-euclide}
\usepackage{lmodern}
\usepackage[normalem]{ulem}
\usetikzlibrary{calc,intersections,patterns}
\tikzset{new/.style={color=red},small label/.style={font=\scriptsize},node label/.style={}}
\newcommand\setItemnumber[1]{\setcounter{enumi}{\numexpr#1-1\relax}}

\DeclareMathOperator{\Pow}{Pow}

\title{Introduction to Euclidean Geometry}

\usetheme{Madrid}
\author{T Yeung}
\institute{THMSS}
\date{2024}
\begin{document}

\setlength{\abovedisplayskip}{3pt}
\setlength{\belowdisplayskip}{3pt}

\frame{\titlepage}
\begin{frame}{Outline}
	\tableofcontents[pausesections]
\end{frame}

\section{Motivation}
\begin{frame}{A brain teaser}
	\begin{columns}
		\column{0.5\textwidth}
		\begin{tikzpicture}
			\coordinate[label=225:$A$] (A) at (0:0);
			\coordinate[label=-90:$B$] (B) at (0:3cm);
			\coordinate[label=40:$C$] (C) at (40:4cm);
			\coordinate[label=120:$D$] (D) at (70:3cm);
			\draw (A) -- (B) -- (C) -- (D) -- cycle;
			\node[inner sep=0.5cm,anchor=230] at (A) {$30^o$};
			\node[inner sep=0.3cm,anchor=-25] at (B) {$50^o$};
			\node[inner sep=0.3cm,anchor=155] at (D) {$40^o$};
			\only<2>{
				\node[inner sep=0.5cm,anchor=195] at (A){$\color{red}40^o$};
				\node[inner sep=0.5cm,anchor=-75] at (B){$\color{red}??$};
				\node[inner sep=0.25cm,anchor=20] at (C){$\color{red}50^o$};
				\node[inner sep=0.35cm,anchor=65] at (C){$\color{red}??$};
				\node[inner sep=0.5cm,anchor=100] at (D){$\color{red}50^o$};
			}
			\draw[name path=AC] (A) -- (C);
			\draw[name path=BD] (B) -- (D);
			\draw[name intersections={of=AC and BD,by=E}] (E) node[label=E]{} ($(E)!5pt!(C)$) -- ([turn]-90:5pt) -- ([turn]-90:5pt);
		\end{tikzpicture}

		\column{0.5\textwidth}
		Given:
		\begin{itemize}
			\item $\angle DAC=30^o$
			\item $\angle CDB=40^o$
			\item $\angle ABD=50^o$
			\item $DB \perp AC$
		\end{itemize}
		What angles can you compute?
	\end{columns}
\end{frame}
\section{Circle}
\subsection{Inscribed Angle Theorem}
\begin{frame}{Inscribed angle theorem}
	\begin{tikzpicture}[radius=1.5cm]
		\coordinate[label=$O$] (O) at (0,0);
		\draw (O) circle;
		\draw (O) -- ++(30:1.5cm) coordinate[label=30:$A$](A);
		\draw (O) -- ++(-50:1.5cm) coordinate[label=300:$B$](B);
		\coordinate[label=170:$C$] (C) at (170:1.5cm);
		\draw (A) -- (C) (B) -- (C);
		\draw[thick,red] ($(O)!0.3cm!(A)$) to[bend left] ($(O)!0.3cm!(B)$);
		\draw[thick,blue] ($(C)!0.3cm!(A)$) to[bend left] ($(C)!0.3cm!(B)$);
	\end{tikzpicture}
	\hfill
	\begin{tikzpicture}[radius=1.5cm]
		\coordinate[label=$O$] (O) at (0,0);
		\draw (O) circle;
		\draw (O) -- ++(30:1.5cm) coordinate[label=30:$A$](A);
		\draw (O) -- ++(-50:1.5cm) coordinate[label=300:$B$](B);
		\coordinate[label=90:$C$] (C) at (90:1.5cm);
		\draw (A) -- (C) (B) -- (C);
		\draw[thick,red] ($(O)!0.3cm!(A)$) to[bend left] ($(O)!0.3cm!(B)$);
		\draw[thick,blue] ($(C)!0.3cm!(A)$) to[bend left] ($(C)!0.3cm!(B)$);
	\end{tikzpicture}
	\hfill
	\begin{tikzpicture}[radius=1.5cm]
			\coordinate[label=$O$] (O) at (0,0);
			\draw (O) circle;
			\draw (O) -- ++(30:1.5cm) coordinate[label=30:$A$](A);
			\draw (O) -- ++(-50:1.5cm) coordinate[label=300:$B$](B);
			\coordinate[label=0:$C$] (C) at (0:1.5cm);
			\draw (A) -- (C) (B) -- (C);
			\draw[thick,red] ($(O)!0.3cm!(A)$) to[bend left] ($(O)!0.3cm!(B)$);
			\draw[thick,blue] ($(C)!0.3cm!(A)$) to[bend left=90] ($(C)!0.3cm!(B)$);
	\end{tikzpicture}
	\begin{minipage}{0.9\textwidth}
		\begin{theorem}[Inscribed Angle Theorem]
			Let $O$ denotes the center of circle, $A$ and $B$ be any two points on the circle, then $\color{red}{\angle AOB} = \color{blue}{2 \angle ACB}$.
		\end{theorem}
	\end{minipage}
\end{frame}
\begin{frame}{Inscribed Angle Theorem (Proof for case I)}
	\begin{columns}
		\column{0.5\textwidth}
		\hfill
		\begin{tikzpicture}[radius=1.5cm,scale=1.5]
			\coordinate[label=$O$] (O) at (0,0);
			\draw (O) circle;
			\draw (O) -- ++(30:1.5cm) coordinate[label=30:$A$](A);
			\draw (O) -- ++(-50:1.5cm) coordinate[label=300:$B$](B);
			\coordinate[label=170:$C$] (C) at (170:1.5cm);
			\draw (A) -- (C) (B) -- (C);
			\draw[thick,green] ($(C)!0.4cm!(A)$) to[bend left,edge label=$a$] ($(C)!0.4cm!(O)$);
			\draw[thick,green] ($(A)!0.4cm!(C)$) to[bend right] node[anchor=20]{$a$} ($(A)!0.4cm!(O)$);
			\draw[thick,magenta] ($(C)!0.4cm!(O)$) to[bend left] node[anchor=160]{$b$} ($(C)!0.4cm!(B)$);
			\draw[thick,magenta] ($(B)!0.4cm!(O)$) to[bend right,edge label'=$b$] ($(B)!0.4cm!(C)$);
			\draw[dashed] (O) -- (C);
		\end{tikzpicture}
		\column{0.5\textwidth}
		Proof:
		\begin{enumerate}
			\item<1-> $OA = OC = OB$	
			\item<2-> $\angle OAC = \angle ACO = a$
			\item<2-> $\angle OCB = \angle OBC = b$
			\item<3-> $\angle ACB = a + b$
			\item<4-> $\angle AOB = 2a + 2b$
		\end{enumerate}
	\end{columns}
\end{frame}
\begin{frame}{Inscribed Angle Theorem (Proof for case II)}
	\begin{columns}
		\column{0.5\textwidth}
		\hfill
		\begin{tikzpicture}[radius=1.5cm, scale=1.5]
			\coordinate[label=$O$,left] (O) at (0,0);
			\draw (O) circle;
			\draw (O) -- ++(30:1.5cm) coordinate[label=30:$A$](A);
			\draw (O) -- ++(-50:1.5cm) coordinate[label=300:$B$](B);
			\coordinate[label=90:$C$] (C) at (90:1.5cm);
			\draw (A) -- (C) (B) -- (C);
			\draw[thick,green] ($(C)!0.4cm!(A)$) to[bend left,edge label=$a$] ($(C)!0.4cm!(O)$);
			\draw[thick,green] ($(A)!0.4cm!(C)$) to[bend right] node[anchor=20]{$a$} ($(A)!0.4cm!(O)$);
			\draw[thick,magenta] ($(C)!0.5cm!(O)$) to[bend right] node[anchor=160]{$b$} ($(C)!0.5cm!(B)$);
			\draw[thick,magenta] ($(B)!0.5cm!(O)$) to[bend left,edge label'=$b$] ($(B)!0.5cm!(C)$);
			\draw[dashed] (O) -- (C);
		\end{tikzpicture}
		\column{0.5\textwidth}
		Proof:
		\begin{enumerate}
			\item<1-> $OA = OC = OB$	
			\item<2-> $\angle OAC = \angle ACO = a$
			\item<2-> $\angle OCB = \angle OBC = b$
			\item<3-> $\angle ACB = a - b$
			\item<4->  
				$
				\begin{aligned}[t]
					\angle AOB &= \angle COB - \angle COA  \\
							   &= (180^o - 2b) \\
							   &- (180^o-2a) \\
							   &= 2a - 2b
				\end{aligned}
				$
		\end{enumerate}
	\end{columns}
\end{frame}
\begin{frame}{Inscribed Angle Theorem (Proof for case III)}
	\begin{columns}
		\column{0.5\textwidth}
		\hfill
		\begin{tikzpicture}[radius=1.5cm,scale=1.5]
				\coordinate[label=$O$,left] (O) at (0,0);
				\draw (O) circle;
				\draw (O) -- ++(30:1.5cm) coordinate[label=30:$A$](A);
				\draw (O) -- ++(-50:1.5cm) coordinate[label=300:$B$](B);
				\coordinate[label=0:$C$] (C) at (0:1.5cm);
				\draw (A) -- (C) (B) -- (C);
				\draw[dashed] (O) -- (C);
				\draw[thick,green] ($(C)!0.3cm!(A)$) to[bend right] node[anchor=-40,inner sep=0cm]{$a$} ($(C)!0.3cm!(O)$);
				\draw[thick,green] ($(A)!0.3cm!(C)$) to[bend left] node[anchor=50,inner sep=0cm]{$a$} ($(A)!0.3cm!(O)$);
				\draw[thick,magenta] ($(C)!0.3cm!(O)$) to[bend right] node[anchor=20,inner sep=0cm]{$b$} ($(C)!0.3cm!(B)$);
				\draw[thick,magenta] ($(B)!0.3cm!(O)$) to[bend left] node[anchor=-80,inner sep=0cm]{$b$} ($(B)!0.3cm!(C)$);
			\end{tikzpicture}
		\column{0.5\textwidth}
		Proof:
		\begin{enumerate}
			\item<1-> $OA = OC = OB$	
			\item<2-> $\angle OAC = \angle ACO = a$
			\item<2-> $\angle OCB = \angle OBC = b$
			\item<3-> $\text{Reflex }\angle ACB = 360^o - a - b$
			\item<4-> $\angle AOB = 360^o - 2a - 2b$
		\end{enumerate}
	\end{columns}
\end{frame}
\begin{frame}{Inscribed Angle Theorem (Corollary I)}
	\begin{center}
		\begin{tikzpicture}[radius=2cm]
			\coordinate (O) at (0,0);
			\draw (O) node[below]{$O$};
			\draw (O) circle;
			\coordinate (A) at (-2,0);
			\coordinate (B) at (2,0);
			\draw (A) node[left]{$A$} -- (B) node[right]{$B$};
			\coordinate (C) at (30:2cm);
			\coordinate (C') at (70:2cm);
			\draw (C) node[above right]{$C$};
			\draw (C') node[above] {$C'$};
			\draw (A) -- (C');
			\draw (B) -- (C');
			\draw (A) -- (C);
			\draw (B) -- (C);
			\draw (A) -- ($(C')!0.2cm!(A)$) -- ([turn]-90:0.2cm) -- ([turn]90:0.2cm);
			\draw (A) -- ($(C)!0.2cm!(A)$) -- ([turn]-90:0.2cm) -- ([turn]90:0.2cm);
		\end{tikzpicture}
	\end{center}
	\begin{center}
		\begin{minipage}{0.9\textwidth}
			\begin{corollary}[$\angle$ in semi circle]
				Let $AB$ be a diameter of circle, $C$ be any point on a circle. Then $\angle ACB = 90^o$ (because the angle at centre is $180^o$).
			\end{corollary}
		\end{minipage}
	\end{center}
\end{frame}
\begin{frame}{Inscribed Angle Theorem (Corollary II)}
	\begin{center}
		\begin{tikzpicture}[radius=2cm]
			\coordinate (O) at (0,0);
			\draw (O) circle;
			\coordinate (A) at (-80:2cm);
			\coordinate (B) at (2,0);
			\draw (A) node[below]{$A$} -- (B) node[right]{$B$};
			\coordinate (C) at (50:2cm);
			\coordinate (C') at (150:2cm);
			\draw (C) node[above right]{$C$};
			\draw (C') node[above] {$C'$};
			\draw (A) -- (C');
			\draw (B) -- (C');
			\draw (A) -- (C);
			\draw (B) -- (C);
			\draw ($(C')!0.2cm!(A)$) to[bend right] ($(C')!0.2cm!(B)$);
			\draw ($(C)!0.2cm!(A)$) to[bend right] ($(C)!0.2cm!(B)$);
		\end{tikzpicture}
	\end{center}
	\begin{center}
		\begin{minipage}{0.9\textwidth}
			\begin{corollary}[angle at circumference $\propto$ arc length]
				The angle subtended by an arc of a circle at the circumference is fixed (because the angle at centre is the same).	
			\end{corollary}
		\end{minipage}
	\end{center}
\end{frame}

\subsection{The Extended Law of Sine}
\begin{frame}{The Extended Law of Sine}
	\begin{columns}
		\column{0.4\textwidth}
			\begin{center}
				\begin{tikzpicture}
					\tkzDefPoints{0/0/A,3/0/B,2/2/C}
					\tkzDefCircle[circum](A,B,C) \tkzGetPoint{O} \tkzGetLength{rayon}
					\tkzDrawCircle[new,dashed,R](O, \rayon pt)
					\tkzDrawPolygon(A,B,C)
					\tkzAutoLabelPoints[center=O](A,B,C)
				\end{tikzpicture}
			\end{center}
		\column{0.6\textwidth}
		\begin{exampleblock}{Naming Convention}
			By convention, in $\Delta ABC$, the opposite side to angle $A$ is named $a$ (similarly for $B$ and $C$), $\mathcal{R}$ denotes the circumradius of $\Delta ABC$, and $r$ denotes the inradius of $\Delta ABC$.
		\end{exampleblock}
		\begin{theorem}[The Extended Law of Sine]
			Given a triangle $ABC$, we have 
			\begin{equation*}
				\frac{a}{\sin A} = \frac{b}{\sin B} = \frac{c}{\sin C} = 2\mathcal{R}
			\end{equation*}
		\end{theorem}
	\end{columns}	
\end{frame}

\begin{frame}{The Extended Law of Sine (Proof)}
	\begin{columns}
		\column{0.4\textwidth}
			\begin{center}
				\begin{tikzpicture}
					\tkzDefPoints{0/0/A,3/0/B,2/2/C}
					\tkzDefCircle[circum](A,B,C) \tkzGetPoint{O} \tkzGetLength{rayon}
					\tkzDrawCircle[new,dashed,R](O, \rayon pt)
					\tkzDrawPolygon(A,B,C)
					\tkzAutoLabelPoints[center=O](A,B,C)
					\onslide<2>{
						\tkzInterLC(B,O)(O,A) \tkzGetPoints{B}{A'}
						\tkzAutoLabelPoints[center=O](A')
						\tkzDrawSegment[new](B,A')
						\tkzDrawSegment[new](A',C)
						\tkzDrawPoint[new](O)
						\tkzLabelPoint[new,above](O){O}
					}
					\onslide<3->{
						\tkzInterLC(B,O)(O,A) \tkzGetPoints{B}{A'}
						\tkzAutoLabelPoints[center=O](A')
						\tkzDrawSegment[](B,A')
						\tkzDrawSegment[](A',C)
						\tkzDrawPoint(O)
						\tkzLabelPoint[above](O){O}
					}
					\onslide<3>{
						\tkzMarkRightAngle[new](B,C,A')
					}
					\onslide<4->{
						\tkzMarkRightAngle[](B,C,A')
					}
				\end{tikzpicture}
			\end{center}
		\column{0.6\textwidth}
		\begin{center}
			\begin{minipage}{0.9\textwidth}
				\begin{theorem}[The Extended Law of Sine]
					Given a triangle $ABC$, we have 
					\begin{equation*}
						\frac{a}{\sin A} = \frac{b}{\sin B} = \frac{c}{\sin C} = 2\mathcal{R}
					\end{equation*}
				\end{theorem}
				Proof:
				\begin{enumerate}
					\item Without loss of generality, we only prove $\frac{a}{\sin A} = 2\mathcal{R}$
					\item<2-> Move $A$ to $A'$ such that $A'B$ is the diameter of the circle.
					\item<3-> Note that $\triangle ACB$ is a right-angled triangle and $\angle BA'C = \angle BAC$.
					\item<4-> We have $\frac{a}{\sin BAC} = \frac{CB}{\angle BA'C} = A'B = 2\mathcal{R}$
				\end{enumerate}
			\end{minipage}
		\end{center}
	\end{columns}	
\end{frame}

\subsection{Relationship between Circumradius and Area}
\begin{frame}{Relationship between Circumradius and Area}
	\begin{columns}
		\column{0.4\textwidth}
		\begin{center}
			\begin{minipage}{0.9\textwidth}
				\begin{tikzpicture}
					\tkzDefPoints{0/0/A,3/0/C,2/2/B}
					\tkzDrawPolygon(A,B,C)
					\tkzDefPointsBy[projection=onto A--C](B){H}
					\tkzDrawSegment(B,H)
					\tkzLabelPoint[left](A){A}
					\tkzLabelPoint[above](B){B}
					\tkzLabelPoint[right](C){C}
					\tkzLabelPoint[below](H){H}
					\tkzLabelLine[pos=0.5,right](B,C){a}
					\tkzLabelLine[pos=0.5,below](A,C){b}
				\end{tikzpicture}
			\end{minipage}
		\end{center}
		\column{0.6\textwidth}
		\begin{minipage}{0.9\textwidth}
			\begin{exampleblock}{Naming Convention}
				$[ABC]$	denotes the area of $ABC$.
			\end{exampleblock}
			\begin{theorem}[Area of a triangle]
				$[ABC] = \frac{1}{2}ab\sin C$.
			\end{theorem}
			Proof:
			\begin{enumerate}
				\item $BH = a\sin C$, $AC=B$
				\item $[ABC] = \frac{1}{2} BH \cdot AC = \frac{1}{2}ab\sin C$.
			\end{enumerate}
			\begin{theorem}[Circumradius and Area]
				By the extended law of sine, we also have $\sin C = \frac{c}{2\mathcal{R}}$, hence, $[ABC] = \frac{abc}{4\mathcal{R}}$
			\end{theorem}
		\end{minipage}
	\end{columns}
\end{frame}

\subsection{Relationship between Circumradius and Side Lengths}
\begin{frame}{Relationship between Circumradius and Side Lengths}
	\begin{center}
		\begin{minipage}{0.9\textwidth}
			\begin{exampleblock}{Naming Convention}
				$s$ denotes the semi-parameter of $\triangle ABC$, i.e. $\frac{a+b+c}{2}$.
			\end{exampleblock}
			We state the Heron's formula without proof:
			\begin{theorem}[Heron's formula]
				In $\triangle ABC$, we have $[ABC] = \sqrt{s(s-a)(s-b)(s-c)}$.
			\end{theorem}
			Together with the result in the previous slide, we can find the circumradius of a triangle if we know all $3$ side lengths:
			\begin{theorem}[Circumradius and Side Lengths]
				\begin{align*}
					[ABC] &= \tfrac{abc}{4\mathcal{R}} \\
					\mathcal{R} &= \tfrac{abc}{4\sqrt{s(s-a)(s-b)(s-c)} }
				\end{align*}
			\end{theorem}
		\end{minipage}
	\end{center}
\end{frame}

\section{Cyclic Quadrilateral}
\begin{frame}{Cyclic Quadrilateral}
	\begin{columns}
		\column{0.4\textwidth}
		\centering
		\begin{tikzpicture}[scale=0.25]
			\tkzDefPoints{0/0/A,12/2/B,9/-5/C}
			\tkzDrawSegments(A,B B,C A,C)
			\tkzDefCircle[circum](A,B,C)
			\tkzGetPoint{O}
			\tkzGetLength{rayon}
			\onslide<2->{
				\tkzDrawCircle[color=red,dashed,R](O,\rayon pt)
			}
		\end{tikzpicture}

		\vspace{1cm}

		\begin{tikzpicture}[scale=0.25]
			\tkzDefPoints{0/0/A,12/4/B,9/8/C,-1/10/D}
			\tkzDrawSegments(A,B B,C C,D D,A)
			\tkzDefCircle[circum](A,B,C)
			\tkzGetPoint{O}
			\tkzGetLength{rayon}
			\onslide<2->{
				\tkzDrawCircle[color=red,dashed,R](O,\rayon pt)
			}
		\end{tikzpicture}

		\column{0.6\textwidth}		
		\begin{center}
			\begin{minipage}{0.9\textwidth}
				\centering
				\begin{block}{Question 1}
					Is it always possible to find a circle passing through a triangle?
				\end{block}
				\begin{block}{Question 2}
					Is it always possible to find a circle passing through a quadrilateral?
				\end{block}
				\pause
				\begin{exampleblock}{Answers}
					Yes, a circumcentre always exists for a triangle;

					No, not possible if a point does not lie on the circumcentre formed by the other three points.
				\end{exampleblock}
			\end{minipage}
		\end{center}
	\end{columns}
\end{frame}
\begin{frame}{Properties for Cyclic Quadrilateral}
	\begin{columns}
		\column{0.4\textwidth}	
		\begin{center}
			\begin{minipage}{0.9\textwidth}
				\begin{center}
					\begin{tikzpicture}[scale=0.35]
						\tkzDefPoints{0/0/O,5/0/A,4/3/B,-4/3/C,-3.535/-3.535/D}
						\tkzDrawPolygon(A,B,C,D)
						\tkzDrawSegments(O,A O,C)
						\tkzDefCircle[circum](A,B,C)
						\tkzGetPoint{O}
						\tkzGetLength{rayon}
						\tkzDrawCircle[R](O,\rayon pt)

						% Label the points
						 \tkzAutoLabelPoints[center=O](A,B,C,D)
						 \tkzLabelPoints(O)

						 % Mark the angle
						 \tkzMarkAngle[color=red](A,O,C)
						 \tkzLabelAngle[pos=1.5,color=red](A,O,C){$2\alpha$}
						 \tkzMarkAngle[color=red](A,D,C)
						 \tkzLabelAngle[pos=1.5,color=red](A,D,C){$\alpha$}
						 \tkzMarkAngle[color=blue](C,O,A)
						 \tkzLabelAngle[pos=1.5,color=blue](C,O,A){$2\beta$}
						 \tkzMarkAngle[color=blue](C,B,A)
						 \tkzLabelAngle[pos=1.5,color=blue](C,B,A){$\beta$}
					\end{tikzpicture}
				\end{center}
			\end{minipage}	
		\end{center}
		\column{0.6\textwidth}
		\begin{center}
			\begin{minipage}{0.9\textwidth}
				\begin{enumerate}
					\item Let $\angle COA = 2\alpha$, $\text{Reflex } \angle COA = 2\beta$
					\pause
					\item $2\alpha + 2\beta = 360^o \implies \alpha + \beta = 180^o$
					\pause
					\item $\angle CDA + \angle CBA = \alpha + \beta = 180^o$
					\pause
				\end{enumerate}
				\begin{theorem}[Supplementary opposite angles]
					Opposite angles inside a cyclic quadrilateral adds up to $180^o$.
				\end{theorem}
				\begin{Corollary}
					Exterior angle equals to the opposite interior angle inside a cyclic quadrilateral.
				\end{Corollary}
			\end{minipage}
		\end{center}
	\end{columns}
\end{frame}
\subsection{Properties for Cyclic Quadrilateral}
\begin{frame}{Properties for Cyclic Quadrilateral}
	\begin{columns}
		\column{0.4\textwidth}	
		\begin{center}
			\begin{minipage}{0.95\textwidth}
					\begin{tikzpicture}[scale=0.35]
						\tkzDefPoints{0/0/O,5/0/A,4/3/B,-4/3/C,-3.535/-3.535/D}
						\tkzDrawPolygon(A,B,C,D)
						\tkzDrawSegments(B,D A,C)
						\tkzDefCircle[circum](A,B,C)
						\tkzGetPoint{O}
						\tkzGetLength{rayon}
						\tkzDrawCircle[R](O,\rayon pt)

						% Label the points
						 \tkzAutoLabelPoints[center=O](A,B,C,D)

						 % Mark the angle
						 \tkzMarkAngle[size=2.3,color=red](A,C,B)
						 \tkzLabelAngle[pos=2.8,color=red](A,C,B){$\alpha$}
						 \tkzMarkAngle[size=2.3,color=red](A,D,B)
						 \tkzLabelAngle[pos=2.8,color=red](A,D,B){$\alpha$}
					\end{tikzpicture}
			\end{minipage}	
		\end{center}

		\column{0.6\textwidth}
		\begin{center}
			\begin{minipage}{0.9\textwidth}
				\begin{theorem}[Angles subtended by the same arc]
					Angles subtended by the same arc are equal.
				\end{theorem}
			\end{minipage}
		\end{center}
	\end{columns}
\end{frame}

\subsection{Test for Cyclic Quadrilateral}
\begin{frame}{Test for Cyclic Quadrilateral}
	\begin{center}
		\begin{minipage}{0.9\textwidth}
			\begin{theorem}[Test for Cyclic Quadrilateral]
				It turns out that the mentioned $3$ properties are also tests for cyclic quadrilateral.
				\begin{itemize}
					\item Opposite angles adds up to $180^o$.
					\item Exterior angle equals the opposite interior angle.
					\item Angles subtended by the same \textbf{side} are equal.
				\end{itemize}
				This means that if \textbf{any} of the above $3$ statement is true, then the quadrilateral is a cyclic quadrilateral.
			\end{theorem}
			The proof is omitted here. Now you should have enough to solve the original problem. :)
		\end{minipage}
		
	\end{center}
\end{frame}

\begin{frame}{Rerouting to our original problem\ldots}
	Now you should have enough to solve the original problem. :)
	\begin{center}
		\begin{tikzpicture}
			\coordinate[label=225:$A$] (A) at (0:0);
			\coordinate[label=-90:$B$] (B) at (0:3cm);
			\coordinate[label=40:$C$] (C) at (40:4cm);
			\coordinate[label=120:$D$] (D) at (70:3cm);
			\draw (A) -- (B) -- (C) -- (D) -- cycle;
			\node[inner sep=0.5cm,anchor=230] at (A) {$30^o$};
			\node[inner sep=0.3cm,anchor=-25] at (B) {$50^o$};
			\node[inner sep=0.3cm,anchor=155] at (D) {$40^o$};
			\node[inner sep=0.5cm,anchor=195] at (A){$\color{red}40^o$};
			\node[inner sep=0.5cm,anchor=-75] at (B){$\color{red}??$};
			\node[inner sep=0.25cm,anchor=20] at (C){$\color{red}50^o$};
			\node[inner sep=0.35cm,anchor=65] at (C){$\color{red}??$};
			\node[inner sep=0.5cm,anchor=100] at (D){$\color{red}50^o$};
			\draw[name path=AC] (A) -- (C);
			\draw[name path=BD] (B) -- (D);
			\draw[name intersections={of=AC and BD,by=E}] (E) node[label=E]{} ($(E)!5pt!(C)$) -- ([turn]-90:5pt) -- ([turn]-90:5pt);
		\end{tikzpicture}
	\end{center}
	Find the remaining angles!
\end{frame}

\section{Power Chord Theorem and its Converse}
\begin{frame}{Power Chord Theorem (I)}
	\begin{columns}
		\column{0.5\textwidth}	
		\begin{minipage}{0.95\textwidth}
			\begin{center}
				\begin{tikzpicture}[scale=0.45]
					\tkzDefPoints{0/0/O,5/0/A,4/3/B,-4/3/C,-3.535/-3.535/D}
					\tkzDrawPolygon(A,B,C,D)
					\tkzDrawSegments(B,D A,C)
					\tkzDefCircle[circum](A,B,C)
					\tkzGetPoint{O}
					\tkzGetLength{rayon}
					\tkzDrawCircle[R](O,\rayon pt)
					\tkzInterLL(A,C)(B,D)
					\tkzGetPoint{P}

					% Label the points
					 \tkzAutoLabelPoints[center=O](A,B,C,D)
					 \tkzLabelPoint[below](P){$P$}
					 \onslide<2->{
						 \tkzMarkAngle[size=1,color=red](D,C,A)
						 \tkzLabelAngle[pos=1.5,color=red](D,C,A){$\alpha$}
						 \tkzMarkAngle[size=1,color=red](D,B,A)
						 \tkzLabelAngle[pos=1.5,color=red](D,B,A){$\alpha$}
						 \tkzMarkAngle[size=1,color=blue](B,D,C)
						 \tkzLabelAngle[pos=1.5,color=blue](B,D,C){$\beta$}
						 \tkzMarkAngle[size=1,color=blue](B,A,C)
						 \tkzLabelAngle[pos=1.5,color=blue](B,A,C){$\beta$}
					 }
				\end{tikzpicture}
			\end{center}
		\end{minipage}	
		\column{0.5\textwidth}
		\begin{enumerate}
			\pause
			\item $\angle DCA = \angle DBA$
			\item $\angle CDB = \angle DAB$
			\pause
			\item $\Delta PCD \sim \Delta PBA$	
			\pause
			\item $\frac{PC}{PB} = \frac{PD}{PA}$
			\pause
			\item $PC \cdot PA = PB \cdot PD$
		\end{enumerate}
	\end{columns}
\end{frame}
\begin{frame}{Power Chord Theorem (I)}
	\begin{columns}
		\column{0.4\textwidth}	
		\begin{minipage}{0.95\textwidth}
			\begin{center}
				\begin{tikzpicture}[scale=0.35]
					\tkzDefPoints{0/0/O,5/0/A,4/3/B,-4/3/C,-2/-4.582/D,0/5/E}
					\tkzDrawSegments(B,D A,C)
					\tkzDefCircle[circum](A,B,C)
					\tkzGetPoint{O}
					\tkzGetLength{rayon}
					\tkzDrawCircle[R](O,\rayon pt)
					\tkzInterLL(A,C)(B,D)
					\tkzGetPoint{P}
					\tkzInterLC(E,P)(O,A)
					\tkzGetPoints{F}{_}
					\tkzDrawSegment(E,F)
					% Label the points
					\tkzAutoLabelPoints[center=O](A,B,C,D,E,F)
					\tkzLabelPoint[below](P){$P$}
					\onslide<2->{
						\tkzLabelPoint[below](O){$O$}
						\tkzDrawPoints(O)
						\tkzInterLC(P,O)(O,A)
						\tkzGetPoints{H}{G}
						\tkzAutoLabelPoints[center=O](G,H)
						\tkzDrawSegments[color=red](G,H)
					}
				\end{tikzpicture}
				\begin{equation*}
					PC \cdot PA = PE \cdot PF = PB \cdot PD
				\end{equation*}
			\end{center}
		\end{minipage}	
		\column{0.6\textwidth}
		\begin{center}
			\begin{minipage}{0.9\textwidth}
				In fact, if we have any chord $XY$ passing through $P$, $PX \cdot PY$ is always fixed.
				\begin{block}{Power of a Point \textbf{inside} the circle}
				$\Pow_\omega(P)$ with respect to the circle $\omega$ is defined to be $|PX \cdot PY|$ for any chord $XY$ passing through $P$.
				\end{block}
				\pause
				\begin{block}{Power = $\mathcal{R}^2-OP^2 > 0$}
					\begin{enumerate}
						\item $\mathcal{R}$ is the radius of circumcircle.
						\pause
						\item Draw a diameter $GH$ through $P$
						\pause
						\item 
							$\begin{aligned}[t]
								\Pow_\omega(P) &= PG \cdot PH \\
											   &= (\mathcal{R}+OP)  \\
											   & \cdot (\mathcal{R}-OP) = \mathcal{R}^2-OP^2
							\end{aligned}$
					\end{enumerate}	
				\end{block}
			\end{minipage}
		\end{center}
	\end{columns}
\end{frame}

\begin{frame}{Power Chord Theorem (II)}
	\begin{columns}
		\column{0.55\textwidth}	
		\begin{minipage}{0.95\textwidth}
			\begin{center}
				\begin{tikzpicture}[scale=0.4]
					\tkzDefPoints{0/0/O,5/0/T,4/3/B,-4/3/C,0/-5/E,7/7/P}
					\tkzDefCircle[circum](T,B,C)
					\tkzGetPoint{O}
					\tkzGetLength{rayon}
					\tkzDrawCircle[R](O,\rayon pt)
					\tkzDefLine[tangent from = P](O,B) \tkzGetPoints{_}{A}
					\tkzDrawSegments(P,C)
					\tkzDrawSegments(P,E)
					\tkzInterLC(C,P)(O,A)
					\tkzGetPoints{B}{C}
					\tkzInterLC(E,P)(O,A)
					\tkzGetPoints{E}{D}
					\tkzAutoLabelPoints[center=O](P,B,C,D,E)
					\tkzDrawSegment[dashed](B,D)
					\tkzDrawSegment[dashed](C,E)
					\onslide<2->{
						 \tkzMarkAngle[size=1,color=red](D,B,P)
						 \tkzLabelAngle[pos=1.5,color=red](D,B,P){$\alpha$}
						 \tkzMarkAngle[size=1,color=red](P,E,C)
						 \tkzLabelAngle[pos=1.5,color=red](P,E,C){$\alpha$}
						 \tkzMarkAngle[size=1,color=blue](P,D,B)
						 \tkzLabelAngle[pos=1.5,color=blue](P,D,B){$\beta$}
						 \tkzMarkAngle[size=1,color=blue](E,C,P)
						 \tkzLabelAngle[pos=1.5,color=blue](E,C,P){$\beta$}
					 }
				\end{tikzpicture}
			\end{center}
		\end{minipage}	
		\column{0.45\textwidth}
		\begin{enumerate}
			\item $\angle PBD = \angle PEC$
			\item $\angle PDB = \angle PCE$
			\item $\Delta PBD \sim \Delta PEC$
			\item $\frac{PB}{PE} = \frac{PD}{PC}$
			\item $PB \cdot PC = PD \cdot PE$
		\end{enumerate}
	\end{columns}
\end{frame}

\begin{frame}{Power Chord Theorem (II)}
	\begin{columns}
		\column{0.4\textwidth}	
		\begin{minipage}{0.95\textwidth}
			\begin{center}
				\begin{tikzpicture}[scale=0.3]
					\tkzDefPoints{0/0/O,5/0/T,4/3/B,-4/3/C,0/-5/E,7/7/P}
					\tkzDefCircle[circum](T,B,C)
					\tkzGetPoint{O}
					\tkzGetLength{rayon}
					\tkzDrawCircle[R](O,\rayon pt)
					\tkzDefLine[tangent from = P](O,B) \tkzGetPoints{_}{A}
					\tkzDrawSegments(P,A)
					\tkzDrawSegments(P,C)
					\tkzDrawSegments(P,E)
					\tkzInterLC(C,P)(O,A)
					\tkzGetPoints{B}{C}
					\tkzInterLC(E,P)(O,A)
					\tkzGetPoints{E}{D}
					\tkzAutoLabelPoints[center=O](P,A,B,C,D,E)
					\onslide<3->{
						\tkzInterLC(O,P)(O,A) \tkzGetPoints{G}{H}
						\tkzDrawSegment[color=red](P,G)
						\tkzDrawPoint(O)
						\tkzAutoLabelPoints[center=O](H,G)
						\tkzLabelPoint[below](O){$O$}
					}
				\end{tikzpicture}
				\begin{equation*}
					PB \cdot PC = PD \cdot PE = PA^2
				\end{equation*}
			\end{center}
		\end{minipage}	
		\column{0.6\textwidth}
		\begin{center}
			\begin{minipage}{0.9\textwidth}
				Again, if we have any chord $XY$ passing through $P$, $PX \cdot PY$ is always fixed.
				\begin{block}{Power of a Point \textbf{outside} the circle}
					$\Pow_\omega(P)$ with respect to the circle $\omega$ is defined to be $-|PX \cdot PY|$ for any chord $XY$ passing through $P$.
				\end{block}
				\pause
				\begin{block}{Power = $\mathcal{R}^2-OP^2 < 0$}
					\begin{enumerate}
						\item Let $R$ be the radius of circumcircle.
							\pause
						\item Draw a diameter $GH$ through $P$
							\pause
						\item 
							$\begin{aligned}[t]
								\Pow_\omega(P) &= -|PG \cdot PH| \\
											   &= (\mathcal{R}+OP)  \\
											   & \cdot (\mathcal{R}-OP) = \mathcal{R}^2-OP^2
							\end{aligned}$
					\end{enumerate}	
				\end{block}
			\end{minipage}
		\end{center}
	\end{columns}
\end{frame}

\begin{frame}{Power Chord Theorem (III)}
	\begin{center}
		\begin{minipage}{0.9\textwidth}
			\begin{block}{Power of a Point \textbf{on} the circle}
				$\Pow_\omega(P)$ with respect to the circle $\omega$ is equal to $0$. (Why does this makes sense?)
			\end{block}
			\begin{block}{Power = $\mathcal{R}^2-OP^2=0$}
				As expected.
			\end{block}
		\end{minipage}
	\end{center}
\end{frame}

\begin{frame}{Converse of Power Chord Theorem}
	\begin{center}
		\begin{minipage}{0.9\textwidth}
			In fact, the converse of power chord theorem is also true.	
			\begin{theorem}[Converse of the Power Chord Theorem]
				Let $A,B,X,Y$ be four distinct points in the plane and let lines  $AB$ and  $XY$ intersect at $P$. Suppose that either $P$ lies in both of the segments $\overline{AB}$ and $\overline{XY}$, or in neither segment. If $PA \cdot PB = PX \cdot PY$, then $A, B, X, Y$ are concyclic.
			\end{theorem}
			This serves as another test for cyclic quadrilateral. The proof is omitted here.
		\end{minipage}
	\end{center}
\end{frame}

\begin{frame}{Another proof of the Pythagoras Theorem}
	\begin{columns}
		\column{0.5\textwidth}
		\begin{minipage}{0.9\textwidth}
			\centering
			\begin{tikzpicture}[scale=0.3]
				\tkzDefPoints{0/0/O,5/0/T,4/3/B,-4/3/C,0/-5/E,7/7/P}
				\tkzDefCircle[circum](T,B,C)
				\tkzGetPoint{O}
				\tkzGetLength{rayon}
				\tkzDrawCircle[R](O,\rayon pt)
				\tkzDefLine[tangent from = P](O,B) \tkzGetPoints{_}{A}
				\tkzDrawSegments(P,O O,A)
				\tkzDrawSegments(P,A)
				\tkzAutoLabelPoints[center=O](P,A)
				\tkzDrawPoint(O)
				\tkzLabelPoint[above](O){O}
			\end{tikzpicture}
		\end{minipage}
		\column{0.5\textwidth}
		\begin{minipage}{0.9\textwidth}
			\begin{block}{Tangent $\perp$ Radius}
				Tangent of a circle at $A$ is perpendicular to $OA$. (Why?)
			\end{block}
			
			\begin{block}{Proof of Pythagoras Theorem}
				Rearranging $\Pow_\omega(P) = PA^2 = OP^2-OA^2$ gives
				\begin{equation*}
					PA^2+OA^2=OP^2
				\end{equation*}
			\end{block}
		\end{minipage}
	\end{columns}
\end{frame}

\section{Properties of Tangent to Circle}
\subsection{Angle in Alternate Segment}
\begin{frame}{A little digression: Angle in Alternate Segment}
	Recall the proof to Power chord theorem (II)
	\begin{columns}
		\column{0.55\textwidth}	
		\begin{minipage}{0.95\textwidth}
			\begin{center}
				\begin{tikzpicture}[scale=0.4]
					\tkzDefPoints{0/0/O,5/0/T,4/3/B,-4/3/C,0/-5/E,7/7/P}
					\tkzDefCircle[circum](T,B,C)
					\tkzGetPoint{O}
					\tkzGetLength{rayon}
					\tkzDrawCircle[R](O,\rayon pt)
					\tkzDefLine[tangent from = P](O,B) \tkzGetPoints{_}{A}
					\tkzInterLC(C,P)(O,A)
					\tkzGetPoints{B}{C}
					\tkzInterLC(E,P)(O,A)
					\tkzGetPoints{E}{D}
					\tkzAutoLabelPoints[center=O](C,E)
					\onslide<-3>{
						\tkzAutoLabelPoints[center=O](B)
						\coordinate[label=60:$P$] (P) at ($(D)!-2cm!(E)$);
						\tkzDrawLine[add=0.3 and 0](D,E)
						\tkzDrawSegment(B,C)
						\tkzDrawSegment(B,D)
						\tkzDrawSegment(C,E)
						\tkzLabelPoint[right](D){$D,\textcolor{red}{B'}$}
					}
					\onslide<-2>{
						\tkzMarkAngle[size=1,color=blue](P,D,B)
						\tkzLabelAngle[pos=1.5,color=blue](P,D,B){$\beta$}
						\tkzMarkAngle[size=1,color=blue](E,C,B)
						\tkzLabelAngle[pos=1.5,color=blue](E,C,B){$\beta$}
					}
					\onslide<2->{
						\tkzDefTangent[at=D](O)
						\tkzGetPoint{h}
						\tkzDrawLine[add=4 and 4,color=red](D,h)
						\tkzDrawSegment[color=red](C,D)
						\coordinate (B'') at ($(D)!5cm!(h)$);
						\tkzLabelPoint[above](B''){$\textcolor{red}{B''}$}
						\coordinate (B''') at ($(h)!5cm!(D)$);
						\tkzLabelPoint[below](B'''){$\textcolor{red}{B'''}$}
					}
					\onslide<3->{
						\tkzDrawSegments[color=red](D,E C,E)
						\tkzMarkAngle[size=1,color=red](E,C,D)
						\tkzLabelAngle[pos=1.5,color=red](E,C,D){$\beta'$}
					}
					\onslide<3>{
						\tkzLabelAngle[pos=1.5,color=red](P,D,B''){$\beta'$}
						\tkzMarkAngle[size=1,color=red](P,D,B'')
					}
					\onslide<4->{
						\tkzMarkAngle[size=1,color=red](E,D,B''')
						\tkzLabelAngle[pos=1.5,color=red](E,D,B'''){$\beta'$}
						\tkzLabelPoint[right](D){$\textcolor{red}{B'}$}
					}
				\end{tikzpicture}
			\end{center}
		\end{minipage}	
		\column{0.45\textwidth}
		\begin{minipage}{0.9\textwidth}
			\begin{block}{Angle in alternate segment}
				\begin{enumerate}
					\item Imagine if $B$ gets increasingly close to $D$ as $B'$.
					\pause
					\item $DB'$ is arbitrarily close to the tangent to the circle at $D$.
					\item Let $B''B'''$ denotes the tangent.
					\pause
					\item $\angle B''DP = \angle DCE$
					\pause
					\item $\angle B'''DE = \angle DCE$
				\end{enumerate}
			\end{block}
		\end{minipage}
	\end{columns}
\end{frame}

\subsection{Other Properties of Tangents}
\begin{frame}{Other Properties of Tangents}
	\begin{columns}
		\column{0.5\textwidth}
		\begin{minipage}{0.9\textwidth}
			\centering
			\begin{tikzpicture}[scale=0.3]
				\tkzDefPoints{0/0/O, 5/0/A, 6/6/P}
				\tkzDrawCircle(O,A)
				\tkzDefTangent[from=P](O,A) \tkzGetPoints{A}{B}
				\tkzDrawSegments[add=0 and .2](P,A P,B)
				\tkzDrawSegments(O,A O,B)
				\tkzDrawPoints(O,A,P,B)
				\tkzAutoLabelPoints[center=O](P,A,B)
				\tkzLabelPoints(O)
				\onslide<3->{
					\tkzDefCircle[circum](O,A,B)
					\tkzGetPoint{O'}
					\tkzGetLength{rayon}
					\tkzDrawCircle[dashed,R,color=red](O',\rayon pt)
				}
			\end{tikzpicture}
			\onslide<2->{
				\begin{equation*}
					\begin{split}
						PA = PB &= OP^2 - \mathcal{R}^2 \\
								&= \Pow_\omega(P)
					\end{split}
				\end{equation*}
			}
		\end{minipage}
		\column{0.5\textwidth}
		\pause
		\begin{minipage}{0.9\textwidth}
			\begin{block}{Equal Tangents}
				Given a point $P$ outside $\omega$  circle, there are two tangents of equal length from $P$ to $\omega$.
			\end{block}
			\pause
			\begin{block}{$OABP$ forms a cyclic quadrilateral}
				$OABP$ is a cyclic quadrilateral since $\angle OBP + \angle OAP = 180^o$. Hence we have $\angle BOA + \angle BPA = 180^o$.
			\end{block}
		\end{minipage}
	\end{columns}
\end{frame}

\section{Example Problems}

\subsection{Prelim 2020 Q6}
\begin{frame}{Example Problem 1}
	\begin{center}
		\begin{minipage}{0.9\textwidth}
			\begin{block}{Question 1}
				(Prelim 2020 Q6) In $\Delta ABC, AB = 6, BC = 7, CA = 8$. Let $D$ be the mid-point of minor arc $AB$ on the circumcentre of $\Delta ABC$. Find $AD^2$.
			\end{block}
			\pause
			Sad news: in math contest, the geometric diagram is usually not provided since \sout{problem setters are lazy} the construction of the diagram is a part of the problem.
			\pause
			\begin{center}
				\begin{tikzpicture}[scale=1.3,rotate=20]
					\tkzDefPoint(0,0){O}
					\tkzDefPoint(220:1){A}
					\tkzDefPoint(140:1){B}
					\tkzDefPoint(20:1){C}
					\tkzDefPoint(180:1){D}
					\tkzAutoLabelPoints[center=O,node label](A,B,C,D)
					\tkzDrawPoints(A,B,C,D)
					\tkzDrawPolygon(A,B,C)
					\tkzDrawCircle(O,A)
					\tkzDefPointBy[projection=onto A--B](D) \tkzGetPoint{H}
				\end{tikzpicture}
			\end{center}
		\end{minipage}
	\end{center}
\end{frame}

\begin{frame}{Example Problem 1 (Solution)}
	\begin{columns}
		\column{0.35\textwidth}
		\begin{center}
			\begin{tikzpicture}[scale=1.6,rotate=20]
				\tkzDefPoint(0,0){O}
				\tkzDefPoint(220:1){A}
				\tkzDefPoint(140:1){B}
				\tkzDefPoint(20:1){C}
				\tkzDefPoint(180:1){D}
				\tkzAutoLabelPoints[center=O,node label](A,B,C,D)
				\tkzDrawPoints(A,B,C,D)
				\tkzDrawPolygon(A,B,C)
				\tkzDrawCircle(O,A)
				\tkzDefPointBy[projection=onto A--B](D) \tkzGetPoint{H}
				\onslide<2>{
					\tkzInterLC(D,H)(O,A) \tkzGetPoints{D}{E}
					\tkzDrawPoints[new](H,O,E)
					\tkzDrawSegment[new](D,E)
					\tkzLabelPoint[new,node label,above](H){H}
					\tkzLabelPoint[new,above](O){O}
					\tkzAutoLabelPoints[new,center=O](E)
				}
				\onslide<3->{
					\tkzInterLC(D,H)(O,A) \tkzGetPoints{D}{E}
					\tkzDrawPoints[](H,O,E)
					\tkzDrawSegment[](D,E)
					\tkzLabelPoint[node label,above](H){H}
					\tkzLabelPoint[above](O){O}
					\tkzAutoLabelPoints[center=O](E)
				}
				\onslide<3>{
				}
			\end{tikzpicture}
		\end{center}
		\column{0.65\textwidth}
		\begin{center}
			\begin{minipage}{0.9\textwidth}
				\begin{itemize}
					\item $AB = 6$, $BC = 7$, $CA=8$
					\item $D$ is the mid-point of $\overset{\frown}{AB}$
					\item Find $AD^2$
				\end{itemize}
				\pause
				\begin{block}{Solution}
					\begin{enumerate}
						\item<2-> Drop a perpendicular line from $D$ to $BA$, that the line passes through $O$.
						\item<3-> $\Pow_\omega(H)= BH \cdot HA = 9$
						\item<4-> Another way to compute $\Pow_\omega$ is $DH \cdot HE$.
						\item<5-> $\mathcal{R}=OD=OE$ is the circumradius of the circle and its given by $\frac{abc}{4\sqrt{s(s-a)(s-b)(s-c)} }$, where $s$ is the semi perimeter.
					\end{enumerate}
				\end{block}
			\end{minipage}
		\end{center}
	\end{columns}
\end{frame}

\begin{frame}{Example Problem 1 (Solution Cont.)}
	\begin{columns}
		\column{0.35\textwidth}
		\begin{center}
			\begin{tikzpicture}[scale=1.6,rotate=20]
				\tkzDefPoint(0,0){O}
				\tkzDefPoint(220:1){A}
				\tkzDefPoint(140:1){B}
				\tkzDefPoint(20:1){C}
				\tkzDefPoint(180:1){D}
				\tkzAutoLabelPoints[center=O,node label](A,B,C,D)
				\tkzDrawPoints(A,B,C,D)
				\tkzDrawPolygon(A,B,C)
				\tkzDrawCircle(O,A)
				\tkzDefPointBy[projection=onto A--B](D) \tkzGetPoint{H}
					\tkzInterLC(D,H)(O,A) \tkzGetPoints{D}{E}
					\tkzDrawPoints[](H,O,E)
					\tkzDrawSegment[](D,E)
					\tkzLabelPoint[node label,above](H){H}
					\tkzLabelPoint[above](O){O}
					\tkzAutoLabelPoints[center=O](E)
			\end{tikzpicture}
		\end{center}
		\column{0.65\textwidth}
		\begin{center}
			\begin{minipage}{0.9\textwidth}
				\begin{block}{Solution}
					\begin{enumerate}
						\setItemnumber{5}
						\item $\mathcal{R} = \frac{6 \cdot 7 \cdot 8}{4\sqrt{\frac{21}{2} \cdot \frac{9}{2} \cdot \frac{7}{2} \cdot \frac{5}{2}} } = \frac{16}{\sqrt{15}}$
						\item 
							$\begin{aligned}[t]
								Pow_\omega (H) &= (OE + OH) \cdot (OD - OH) \\
											   &= (\tfrac{16}{\sqrt{15}})^2 - OH^2 \\
											   &= 9
							\end{aligned}$
							$OH = \frac{11}{\sqrt{15}}$
						\item 
							$\begin{aligned}[t]
								AD^2 &= DH^2 + AH^2 \\
									 &= (\tfrac{5}{\sqrt{15}})^2 + 3^2 \\
									 &= \tfrac{32}{3}
							\end{aligned}$
					\end{enumerate}
				\end{block}
			\end{minipage}
		\end{center}
	\end{columns}
\end{frame}

\subsection{Prelim 2022 Q16}
\begin{frame}{Example Problem 2}
	\begin{center}
		\begin{minipage}{0.9\textwidth}
			\begin{block}{Question 2}
				(Prelim 2022 Q16) $ABCD$ is a parallelogram with $\angle B$ acute. A circle is tangent to $BC$, $CD$ and $DA$. The circle intersects $AC$ at $M$ and $N$, where $M$ is closer to $A$ than $N$. If $AM = 9$, $MN = 16$ and $NC = 2$, find the area of $ABCD$.
			\end{block}
			\pause
			\begin{center}
				\begin{tikzpicture}[scale=1]
					\tkzDefPoint(0,0){O}
					\tkzDefPoint(-30:1){R}
					\tkzDefPoint(150:1){S}
					\tkzDefPoint(-90:1){T}
					\tkzDrawCircle(O,R)
					\tkzDefLine[orthogonal=through R](O,R) \tkzGetPoint{X}
					\tkzDefLine[orthogonal=through T](O,T) \tkzGetPoint{Y}
					\tkzDefLine[orthogonal=through S](O,S) \tkzGetPoint{Z}
					\tkzInterLL(S,Z)(T,Y) \tkzGetPoint{D}
					\tkzInterLL(T,Y)(R,X) \tkzGetPoint{C}
					\tkzDrawSegment(C,D)
					\draw (D) -- ($(D)!3cm!(S)$) coordinate(A);
					\draw (C) -- ($(C)!3cm!(R)$) coordinate(B);
					\tkzDrawSegment(B,A)
					\tkzDrawSegment(A,C)
					\tkzInterLC(A,C)(O,R) \tkzGetPoints{N}{M}
					\tkzLabelPoint[above](N){N}
					\tkzLabelPoint[](M){M}
					\tkzAutoLabelPoints[center=O](A,B,C,D)	
				\end{tikzpicture}
			\end{center}
		\end{minipage}
	\end{center}
\end{frame}

\begin{frame}{Example Problem 2 (Solution)}
	\begin{columns}
		\column{0.4\textwidth}
			\begin{center}
				\begin{tikzpicture}[scale=0.9]
					\tkzDefPoint(0,0){O}
					\tkzDefPoint(-30:1){R}
					\tkzDefPoint(150:1){S}
					\tkzDefPoint(-90:1){T}
					\tkzDrawCircle(O,R)
					\tkzDefLine[orthogonal=through R](O,R) \tkzGetPoint{X}
					\tkzDefLine[orthogonal=through T](O,T) \tkzGetPoint{Y}
					\tkzDefLine[orthogonal=through S](O,S) \tkzGetPoint{Z}
					\tkzInterLL(S,Z)(T,Y) \tkzGetPoint{D}
					\tkzInterLL(T,Y)(R,X) \tkzGetPoint{C}
					\tkzDrawSegment(C,D)
					\draw (D) -- ($(D)!3cm!(S)$) coordinate(A);
					\draw (C) -- ($(C)!3cm!(R)$) coordinate(B);
					\tkzDrawSegment(B,A)
					\tkzDrawSegment(A,C)
					\tkzInterLC(A,C)(O,R) \tkzGetPoints{N}{M}
					\tkzLabelPoint[above,inner sep=2mm,small label](N){N}
					\tkzLabelPoint[small label](M){M}
					\tkzAutoLabelPoints[center=O,small label](A,B,C,D)	
					\tkzLabelPoint[below,small label](T){T}
					\tkzLabelPoint[left,small label](S){S}
					\tkzLabelPoint[right,small label](R){R}
					\onslide<4>{
						\begin{scope}[new]
							\tkzDefPointBy[projection=onto A--D](C) \tkzGetPoint{H}
							\tkzLabelPoint[small label,left,new](H){H}
							\tkzFillPolygon[color=teal!20,opacity=0.3](H,S,R,C)
						\end{scope}
					}
					\onslide<5->{
						\tkzDefPointBy[projection=onto A--D](C) \tkzGetPoint{H}
						\tkzLabelPoint[small label,left](H){H}
						\tkzFillPolygon[color=teal!20,opacity=0.3](H,S,R,C)
					}
				\end{tikzpicture}
			\end{center}
		\column{0.6\textwidth}
		\begin{minipage}{0.9\textwidth}
			\begin{itemize}
				\item $AM=9,MN=16,NC=12$
				\item Circle $\omega$ tangent to $BC$, $CD$, $DA$ at $R,S,T$
			\end{itemize}
			\pause
			\begin{block}{Solution}
				\begin{enumerate}
					\item $\Pow_\omega(A) = -AM \cdot AN = -AS^2 \Rightarrow AS = 15$
					\pause
					\item $\Pow_\omega(C) = -CN \cdot CM = -CT^2 \Rightarrow CT = CR = 6$
					\pause
					\item Drop a perpendicular line from $C$ to $AD$ at $H$, note that $SRCH$ forms a rectangle. Hence $SH=CR=6$
					\pause
					\item Let $SD=DT=x$, in  $\triangle DHC$, by Pythagoras Theorem, $(x-6)^2+(x+6)^2=AC^2-AH^2$
				\end{enumerate}
			\end{block}
		\end{minipage}
	\end{columns}	
\end{frame}

\begin{frame}{Example Problem 2 (Solution Cont.)}
	\begin{columns}
		\column{0.4\textwidth}
			\begin{center}
				\begin{tikzpicture}[scale=0.9]
					\tkzDefPoint(0,0){O}
					\tkzDefPoint(-30:1){R}
					\tkzDefPoint(150:1){S}
					\tkzDefPoint(-90:1){T}
					\tkzDrawCircle(O,R)
					\tkzDefLine[orthogonal=through R](O,R) \tkzGetPoint{X}
					\tkzDefLine[orthogonal=through T](O,T) \tkzGetPoint{Y}
					\tkzDefLine[orthogonal=through S](O,S) \tkzGetPoint{Z}
					\tkzInterLL(S,Z)(T,Y) \tkzGetPoint{D}
					\tkzInterLL(T,Y)(R,X) \tkzGetPoint{C}
					\tkzDrawSegment(C,D)
					\draw (D) -- ($(D)!3cm!(S)$) coordinate(A);
					\draw (C) -- ($(C)!3cm!(R)$) coordinate(B);
					\tkzDrawSegment(B,A)
					\tkzDrawSegment(A,C)
					\tkzInterLC(A,C)(O,R) \tkzGetPoints{N}{M}
					\tkzLabelPoint[above,inner sep=2mm,small label](N){N}
					\tkzLabelPoint[small label](M){M}
					\tkzAutoLabelPoints[center=O,small label](A,B,C,D)	
					\tkzLabelPoint[below,small label](T){T}
					\tkzLabelPoint[left,small label](S){S}
					\tkzLabelPoint[right,small label](R){R}
					\tkzDefPointBy[projection=onto A--D](C) \tkzGetPoint{H}
					\tkzLabelPoint[small label,left](H){H}
					\tkzFillPolygon[color=teal!20,opacity=0.3](H,S,R,C)
				\end{tikzpicture}
			\end{center}
		\column{0.6\textwidth}
		\begin{center}
			\begin{minipage}{0.9\textwidth}
				\begin{itemize}
				\item $AM=9,MN=16,NC=12,AS=15,CT=CR=SH=6$
				\end{itemize}
				\begin{block}{Solution}
					\begin{enumerate}
						\setItemnumber{5}
						\item Solving $(x-6)^2-(x+6)^2=AC^2-AH^2=27^2-21^2=288$ gives $x=12$. 
						\pause
						\item $[ABCD] = AD \cdot CH = (12+15)\sqrt{288} = 324\sqrt{2} $
					\end{enumerate}
				\end{block}
				\pause
				\begin{block}{Question}
					Where did we use the parallelogram condition?	
				\end{block}
			\end{minipage}
		\end{center}
	\end{columns}	
\end{frame}

\section{Practice Problems}
\begin{frame}{Practice Problems}
	\begin{block}{Question 1}
		(Prelim 2023 Q17) $ABCD$ is a square. $P$ is a point inside $ABCD$ such that $\angle APD + \angle BPC = 180^o$ and  $\angle BPC$ is acute.  If $PB = 3$ and $PC = 4$, find $BC$.
	\end{block}
	\begin{block}{Question 2}
		(Prelim 2021 Q12) $OABC$ is a trapezium with $OC \parallel AB$ and $\angle AOB = 37^o$	. Furthermore, $A,B,C$ all lie on the circumference of a circle centered at $O$. The perpendicular bisector of $OC$ meets $AC$ at $D$.
	\end{block}
	\begin{block}{Question 3}
		(Prelim 2018 Q13) Let $O$ be the circumcentre of $\triangle ABC$. Suppose $AB = 1$ and $AO = AC = 2$. $D$ and $E$ are points on the extensions of $AB$ and $A C$ respectively such that $OD = OE $ and $BD = \sqrt{2} EC$. Find the value of $OD^2$.
	\end{block}
\end{frame}
\begin{frame}{}
  \centering \Large
  \emph{Thank You!}
\end{frame}
\end{document}
