\documentclass{beamer}
\usepackage[utf8]{inputenc}
\usepackage{xcolor}
\usepackage{tikz}
\usepackage{tkz-euclide}
\usepackage{lmodern}
\usepackage[normalem]{ulem}
\usetikzlibrary{calc,intersections,patterns}
\tikzset{new/.style={color=red},small label/.style={font=\scriptsize},node label/.style={}}
\newcommand\setItemnumber[1]{\setcounter{enumi}{\numexpr#1-1\relax}}

\DeclareMathOperator{\Pow}{Pow}

\title{Length and Ratio}

\usetheme{Madrid}
\author{T Yeung}
\institute{THMSS}
\date{2024}
\begin{document}

\setlength{\abovedisplayskip}{3pt}
\setlength{\belowdisplayskip}{3pt}

\frame{\titlepage}
\begin{frame}{Outline}
	\tableofcontents[pausesections]
\end{frame}

\section{Trigonometric Identities}
\begin{frame}{Trigonometric Identities}
	Trigonometric identities simplify complex expressions.
	\begin{block}{}
		\begin{equation*}
			\begin{split}
				1 &= \sin^2 \theta + \cos^2 \theta \\
				\sin(-\theta) &= - \sin \theta \\
				\cos(-\theta) &= \cos \theta \\
				\sin(\alpha + \beta) &= \sin \alpha \cos \beta + \sin \beta \cos \alpha \\ 
				\cos(\alpha + \beta) &= \cos \alpha \cos \beta - \sin \alpha \sin \beta \\
			\end{split}
		\end{equation*}
	\end{block}
	We also have the product-to-sum identities
	\begin{block}{}
		\begin{equation*}
			\begin{split}
				2\cos \alpha \cos \beta &= \cos(\alpha - \beta) + \cos(\alpha + \beta) \\
				2\sin \alpha \sin \beta &= \cos(\alpha - \beta) - \cos(\alpha + \beta) \\
				2\sin \alpha \cos \beta &= \sin(\alpha - \beta) + \sin(\alpha + \beta)
			\end{split}
		\end{equation*}	
	\end{block}
	that directly follows from the expansion of compound angle formula.
\end{frame}
\begin{frame}{Trigonometric Identities (cont)}
	By substituting $\frac{\alpha+\beta}{2}$ in $\alpha$ and $\frac{\alpha - \beta}{2}$ in $\beta$, we have the sum-to-product identities.
	\begin{block}{}
		\begin{equation*}
			\begin{split}
				\cos (a) \cos (b)&=\frac{1}{2}(\cos (a+b)+\cos (a-b) \\
				\sin (a) \sin (b)&=\frac{1}{2}(\cos (a-b)-\cos (a+b)) \\
				\sin (a) \cos (b)&=\frac{1}{2}(\sin (a+b)+\sin (a-b)) \\
			\end{split}
		\end{equation*}
	\end{block}
\end{frame}
\section{The Extended Law of Sine}

\begin{frame}{The Extended Law of Sine}
	Recall the Extended Law of Sine we derived:
	\begin{theorem}[The Extended Law of Sine]
		Given a triangle $ABC$, we have 
		\begin{equation*}
			\frac{a}{\sin A} = \frac{b}{\sin B} = \frac{c}{\sin C} = 2\mathcal{R}
		\end{equation*}
	\end{theorem}
	We can use this result to prove the Ptolemy's Theorem.
\end{frame}

\begin{frame}{Ptolemy's Theorem}
	\begin{theorem}[Ptolemy's Theorem]
		Let $ABCD$ be a cyclic quadrilateral. Then
		\begin{equation*}
			AB \cdot CD + BC \cdot DA = AC \cdot BD
		\end{equation*}
	\end{theorem}
	\begin{figure}
		\centering
		\begin{tikzpicture}[small label]
			\def\r{1.8}
			\tkzDefPoint(0,0){O}
			\tkzDefPoint(105:\r){A}
			\tkzDefPoint(60:\r){D}
			\tkzDefPoint(-60:\r){C}
			\tkzDefPoint(180:\r){B}
			\tkzDrawCircle(O,A)
			\tkzDrawPolygon(A,B,C,D)
			\tkzDrawSegments(A,C B,D)
			\tkzAutoLabelPoints[center=O](A,B,C,D)
			% Angle
		\end{tikzpicture}
	\end{figure}
\end{frame}

\begin{frame}{Ptolemy's Theorem (proof)}
	\begin{columns}
		\column{0.5\textwidth}
		\begin{figure}
			\centering
			\begin{tikzpicture}[small label]
				\def\r{1.8}
				\tkzDefPoint(0,0){O}
				\tkzDefPoint(125:\r){A}
				\tkzDefPoint(60:\r){D}
				\tkzDefPoint(-40:\r){C}
				\tkzDefPoint(190:\r){B}
				\tkzDrawCircle(O,A)
				\tkzDrawPolygon(A,B,C,D)
				\tkzDrawSegments(A,C B,D)
				\tkzAutoLabelPoints[center=O](A,B,C,D)
				\onslide<3> {
					\tkzLabelAngle[new,small label,pos=0.5](A,D,B){$\alpha_1$}
					\tkzLabelAngle[new,small label,pos=0.5](B,A,C){$\alpha_2$}
					\tkzLabelAngle[new,small label,pos=0.5](C,B,D){$\alpha_3$}
					\tkzLabelAngle[new,small label,pos=0.5](D,C,A){$\alpha_4$}
				}
				\onslide<4-> {
					\tkzLabelAngle[small label,pos=0.5](A,D,B){$\alpha_1$}
					\tkzLabelAngle[small label,pos=0.5](B,A,C){$\alpha_2$}
					\tkzLabelAngle[small label,pos=0.5](C,B,D){$\alpha_3$}
					\tkzLabelAngle[small label,pos=0.5](D,C,A){$\alpha_4$}
				}
				% Angle
			\end{tikzpicture}
		\end{figure}
		\column{0.5\textwidth}
		\begin{enumerate}
			\item WLOG, we set $\mathcal{R} = \frac{1}{2}$ as the radius of $(ABCD)$
				\pause
			\item Note $AB = \sin \angle AXB$ for any point $X$ on the circumcircle.
				\pause
			\item A reasonable choice for our parameters is $\angle ADB, \angle BAC, \angle CBD, \angle DCA$.
				\pause
			\item They sum up to $180^o$
				\pause
			\item $AB = \sin \alpha_1$, $BC=\sin \alpha_2$, $CD=\sin \alpha_3$,  $DA=\sin \alpha_4$
				\pause
			\item  $AC = \sin \angle ABC = \sin (\alpha_3 + \alpha_4)$,
				\pause
				$BD = \sin \angle DAB = \sin(\alpha_2 + \alpha_3)$
				\pause
			\item Now what we want to show is $\sin \alpha_1 \sin \alpha_3 + \sin \alpha_2 \sin \alpha_4 = \sin(\alpha_3 + \alpha_4)\sin(\alpha_2 + \alpha_3)$
		\end{enumerate}
	\end{columns}
\end{frame}

\begin{frame}{Ptolemy's Theorem (proof cont.)}
	\begin{enumerate}
		\setItemnumber{8}
		\item Note that by product-to-sum identities, we have
			\begin{equation*}
				\begin{split}
					\sin \alpha_1 \sin \alpha_3 &= \frac{1}{2}(\cos(\alpha_1	- \alpha_3) - \cos(\alpha_1 + \alpha_3)) \\
					\sin \alpha_2 \sin \alpha_4 &= \frac{1}{2}(\cos(\alpha_2-\alpha_4)-\cos(\alpha_2+\alpha_4)) \\
					\sin(\alpha_2+\alpha_3)\sin(\alpha_3+\alpha_4) &= \frac{1}{2}(\cos(\alpha_2-\alpha_4) - \cos(\alpha_2+2\alpha_3+\alpha_4))
				\end{split}
			\end{equation*}
			\pause
		\item Since $\alpha_1 + \alpha_2 + \alpha_3 + \alpha_4 = 180^o$, we also have 
			\begin{equation*}
				\cos(\alpha_1+\alpha_3)+\cos(\alpha_2+\alpha_4)=0
			\end{equation*}
			\pause
		\item Also note that
			\begin{equation*}
				\cos(\alpha_2+2\alpha_3+\alpha_4) = \cos(180^o-\alpha_1+\alpha_3)=-\cos(\alpha_1-\alpha_3)
			\end{equation*}
		\pause
		\item The rest is trivial. :)
	\end{enumerate}		
\end{frame}

\begin{frame}{Stewart's Theorem}
	\begin{corollary}[Stewart's Theorem]
		Let $ABC$ be a triangle. Let $D$ be a point on $\overline{BC}$ and let $m = DB$, $n = DC$, $d = AD$. Then
		\begin{equation*}
			a(d^2+mn)=b^2m+c^2n
		\end{equation*}
	\end{corollary}
	Often this is written in the form
	\begin{equation*}
		man + dad = bmb + cnc
	\end{equation*}
	as a mnemonic - "a \emph{man} and his \emph{dad} put a \emph{bomb} in the \emph{sink}".
	\begin{figure}
		\centering
		\begin{tikzpicture}[small label]
			\def\r{1.3}
			\tkzDefPoint(0,0){O}
			\tkzDefPoint(120:\r){A}
			\tkzDefPoint(210:\r){B}
			\tkzDefPoint(-30:\r){C}

			\tkzDefBarycentricPoint(B=1,C=2) \tkzGetPoint{D}
			\tkzDrawSegment(A,D)
			\tkzAutoLabelPoints[center=O](A,B,C,D)
			\tkzDrawPolygon(A,B,C)
			\tkzDrawCircle(O,A)
			\tkzLabelLine[left](A,B){c}
			\tkzLabelLine[right](A,D){d}
			\tkzLabelLine[right](A,C){b}
			\tkzLabelLine[above](B,D){m}
			\tkzLabelLine[above](D,C){n}
		\end{tikzpicture}
	\end{figure}
\end{frame}

\begin{frame}{Stewart's Theorem (proof)}
	\begin{columns}
		\column{0.5\textwidth}
		\begin{figure}
			\centering
			\begin{tikzpicture}[small label]
				\def\r{1.7}
				\tkzDefPoint(0,0){O}
				\tkzDefPoint(120:\r){A}
				\tkzDefPoint(210:\r){B}
				\tkzDefPoint(-30:\r){C}

				\tkzDefBarycentricPoint(B=1,C=2) \tkzGetPoint{D}
				\tkzInterLC(A,D)(O,A) \tkzGetSecondPoint{P}
				\tkzDrawSegment(A,D)
				\tkzDrawSegments[dashed](C,P D,P B,P)
				\tkzDrawCircle(O,A)
				\tkzDrawPolygon(A,B,C)
				\tkzAutoLabelPoints[center=O](A,B,C,P)
				\tkzLabelPoint[small label,above](D){D}
				\tkzLabelLine[left](A,B){c}
				\tkzLabelLine[right](A,D){d}
				\tkzLabelLine[right](A,C){b}
				\tkzLabelLine[above](B,D){m}
				\tkzLabelLine[above](D,C){n}
			\end{tikzpicture}
		\end{figure}
		\column{0.5\textwidth}
		\begin{enumerate}
			\item Let $AD$ meet $(ABC)$ again at $P$.
				\pause
			\item By similar triangle, we have $\frac{BP}{m} = \frac{b}{d}$ and $\frac{CP}{n} = \frac{c}{d}$
				\pause
			\item By Power Chord Theorem, we know that $DP = \frac{mn}{d}$
				\pause
			\item By Ptolemy's Theorem, we have
				$BC \cdot AP = AC \cdot BP + AB \cdot CP$
				\pause
			\item Hence,  $a\cdot (d+\frac{mn}{d}) = b\cdot \frac{bm}{d} + c \cdot \frac{cn}{d}$ which is the Stewart's Theorem.
		\end{enumerate}
	\end{columns}
\end{frame}

\begin{frame}{Prelim 2019 Q13}
	Another application of the Extended Law of Sine:
	\begin{block}{Prelim 2019 Q13}
		$A,B,C$ are three points on a circle while $P$ and $Q$ are two points on $AB$. The extensions of $CP$ and $CQ$ meet the circle at $S$ and $T$ respectively. If $AP=2$,$AQ=7$, $AB=11$,$AS=5$ and $BT=2$, find the length of $ST$.
	\end{block}
	\pause
	\begin{figure}
		\centering
		\begin{tikzpicture}[small label]
			\def\r{1.6}
			\tkzDefPoint(0,0){O}
			\tkzDefPoint(180:\r){A}
			\tkzDefPoint(30:\r){B}
			\tkzDefPoint(110:\r){C}
			\tkzDefPoint(230:\r){S}
			\tkzDefPoint(-0:\r){T}
			\tkzDrawCircle(O,A)
			\tkzDrawPolygon(A,B,C)
			\tkzDrawSegment(C,S)
			\tkzDrawSegment(C,T)
			\tkzDrawSegment(A,S)
			\tkzDrawSegment(B,T)
			\tkzInterLL(A,B)(C,S) \tkzGetPoint{P}
			\tkzInterLL(A,B)(C,T) \tkzGetPoint{Q}
			\tkzAutoLabelPoints[center=O](A,B,C,S,T)
			\tkzLabelPoint[small label,below right](P){P}
			\tkzLabelPoint[small label,below](Q){Q}
			\tkzLabelLine[above](A,P){2}
			\tkzLabelLine[above](P,Q){5}
			\tkzLabelLine[pos=0.4,above](Q,B){4}
			\tkzLabelLine[left](A,S){5}
			\tkzLabelLine[right](B,T){2}
		\end{tikzpicture}
	\end{figure}
	\pause
	The ratio is not easy to get right\ldots
\end{frame}

\begin{frame}{Prelim 2019 Q13 (Solution)}
	\begin{columns}
		\column{0.5\textwidth}
		\begin{figure}
			\centering
			\begin{tikzpicture}[small label]
				\def\r{1.9}
				\tkzDefPoint(0,0){O}
				\tkzDefPoint(180:\r){A}
				\tkzDefPoint(30:\r){B}
				\tkzDefPoint(110:\r){C}
				\tkzDefPoint(230:\r){S}
				\tkzDefPoint(-0:\r){T}
				\tkzDrawCircle(O,A)
				\tkzDrawPolygon(A,B,C)
				\tkzDrawSegment(C,S)
				\tkzDrawSegment(C,T)
				\tkzDrawSegment(A,S)
				\tkzDrawSegment(B,T)
				\tkzInterLL(A,B)(C,S) \tkzGetPoint{P}
				\tkzInterLL(A,B)(C,T) \tkzGetPoint{Q}
				\tkzAutoLabelPoints[center=O](A,B,C,S,T)
				\tkzLabelPoint[small label,below right](P){P}
				\tkzLabelPoint[small label,below](Q){Q}
				\tkzLabelLine[above](A,P){2}
				\tkzLabelLine[above](P,Q){5}
				\tkzLabelLine[pos=0.4,above](Q,B){4}
				\tkzLabelLine[left](A,S){5}
				\tkzLabelLine[right](B,T){2}
			\end{tikzpicture}
		\end{figure}
		\column{0.5\textwidth}
		\pause
		\begin{enumerate}
			\item The key observation is that $\frac{AS}{\sin \angle ACS} = \frac{ST}{\sin \angle SCT}$
				\pause
			\item It suffices to find $\frac{\sin \angle ACS}{\sin \angle SCT}$.
				\pause
			\item Consider $\triangle ACQ$, we know that $\frac{QC}{AC} = 2$
				\pause
			\item We also have $\frac{PQ}{\sin \angle PCQ} = \frac{QC}{\sin \angle CPQ} = \frac{QC}{\sin \angle CPA} =  \frac{QC}{AC} \cdot \frac{AC}{\sin \angle CPA} = 2 \cdot \frac{AP}{\sin \angle ACP}$
				\pause
			\item Equating the expressions at the ends, we have $\frac{\sin \angle ACP}{\sin \angle PCQ}  = 2 \cdot \frac{AP}{PQ} = \frac{4}{5}$
				\pause
			\item Using the first result, we have $ST = \frac{25}{4}$
		\end{enumerate}
	\end{columns}	
\end{frame}

\section{Angle Bisector Theorem}
\begin{frame}{Angle Bisector Theorem}
	The third application of the Extended Law of Sine is in proving the Angle Bisector Theorem.	 In fact, the derivation is exactly the same as a part of the previous question. Recall in $\triangle AQC$, we derived that $\frac{\sin \angle ACP}{\sin \angle PCQ} = \frac{QC}{AC} \cdot \frac{AP}{PQ}$. Rearranging this gives 
	$$AC \cdot PQ \cdot \sin \angle ACP = QC \cdot AP \cdot \sin \angle PCQ$$
	which is generally true.
	\begin{figure}
		\centering
		\begin{tikzpicture}[small label]
			\def\r{1.9}
			\tkzDefPoint(0,0){O}
			\tkzDefPoint(180:\r){A}
			\tkzDefPoint(30:\r){B}
			\tkzDefPoint(110:\r){C}
			\tkzDrawPolygon(A,C,Q)
			\tkzInterLL(A,B)(C,S) \tkzGetPoint{P}
			\tkzInterLL(A,B)(C,T) \tkzGetPoint{Q}
			\tkzDrawSegment(C,P)
			\tkzAutoLabelPoints[center=O](A,C)
			\tkzLabelPoint[small label,below](P){P}
			\tkzLabelPoint[small label,below](Q){Q}
		\end{tikzpicture}
	\end{figure}
	When $CP$ is an angle bisector, we have the angle bisector theorem.
\end{frame}

\begin{frame}{Angle Bisector Theorem}
	The Angle Bisector Theorem is a special case of the above equality.
	\begin{theorem}[Angle Bisector Theorem]
		Let $BD$ be the angle bisector of $\angle ABC$, then $AB:BC = AD:DC$.
	\end{theorem}
	\begin{figure}
		\centering	
		\begin{tikzpicture}
			\def\r{1.6}
			\tkzDefPoint(0,0){O}
			\tkzDefPoint(180:\r){A}
			\tkzDefPoint(80:\r){B}
			\tkzDefPoint(0:\r){C}
			\tkzDrawPolygon(A,B,C)
			\tkzDefLine[bisector](A,B,C) \tkzGetPoint{b}
			\tkzInterLL(B,b)(A,C) \tkzGetPoint{D}
			\tkzDrawSegment(B,D)
			\tkzAutoLabelPoints[center=O](A,B,C)
			\tkzLabelPoint[below](D){D}
			\tkzMarkAngle[size=0.5,cyan,mark=|](A,B,D)
			\tkzMarkAngle[size=0.6,cyan,mark=|](D,B,C)
		\end{tikzpicture}
	\end{figure}
\end{frame}

\begin{frame}{Prelim 2022 Q20}
	\begin{block}{Prelim 2022 Q20}
		Let $ABCD$ be a cyclic quadrilateral and $E$ be the intersection of $AC$ and $BD$. $P$ and $Q$ are two points on $AC$ such that the points $A,E,Q,P,C$ lie on the same straight line in this order, and that $BP$ bisects $\angle ABC$ whereas $DQ$ bisects $\angle ADC$. If $AE=4$,$EQ=2$, and $QP=3$, find the length of $PC$.
	\end{block}
	\pause
	\begin{figure}
		\centering
		\begin{tikzpicture}[small label]
			\def\r{1.8}
			\tkzDefPoint(0,0){O}
			\tkzDefPoint(105:\r){A}
			\tkzDefPoint(60:\r){D}
			\tkzDefPoint(-60:\r){C}
			\tkzDefPoint(180:\r){B}
			\tkzDrawCircle(O,A)
			\tkzDrawPolygon(A,B,C,D)
			\tkzDrawSegments(A,C B,D)
			\tkzAutoLabelPoints[center=O](A,B,C,D)
			% Intersection
			\tkzInterLL(A,C)(B,D) \tkzGetPoint{E}
			\tkzLabelPoint[small label,above](E){E}
			% Bisector
			\tkzDefLine[bisector](A,B,C) \tkzGetPoint{b}
			\tkzInterLL(B,b)(A,C) \tkzGetPoint{P}
			\tkzDefLine[bisector](C,D,A) \tkzGetPoint{d}
			\tkzInterLL(D,d)(A,C) \tkzGetPoint{Q}
			\tkzDrawSegments(B,P D,Q)
			\tkzLabelPoint[small label,right](P){P}
			\tkzLabelPoint[small label,below left](Q){Q}
			\tkzMarkAngle[orange,mark=|,size=0.5](A,D,Q)
			\tkzMarkAngle[orange,mark=|,size=0.6](Q,D,C)
			\tkzMarkAngle[cyan,mark=||,size=0.5](P,B,A)
			\tkzMarkAngle[cyan,mark=||,size=0.6](C,B,P)
		\end{tikzpicture}
	\end{figure}
\end{frame}

\begin{frame}{Prelim 2022 Q20 Solution}
	\begin{columns}
		\column{0.5\textwidth}
		\begin{figure}
			\centering
			\begin{tikzpicture}[small label]
				\def\r{2}
				\tkzDefPoint(0,0){O}
				\tkzDefPoint(105:\r){A}
				\tkzDefPoint(60:\r){D}
				\tkzDefPoint(-60:\r){C}
				\tkzDefPoint(180:\r){B}
				\tkzDrawCircle(O,A)
				\tkzDrawPolygon(A,B,C,D)
				\tkzDrawSegments(A,C B,D)
				\tkzAutoLabelPoints[center=O](A,B,C,D)
				% Intersection
				\tkzInterLL(A,C)(B,D) \tkzGetPoint{E}
				\tkzLabelPoint[small label,above](E){E}
				% Bisector
				\tkzDefLine[bisector](A,B,C) \tkzGetPoint{b}
				\tkzInterLL(B,b)(A,C) \tkzGetPoint{P}
				\tkzDefLine[bisector](C,D,A) \tkzGetPoint{d}
				\tkzInterLL(D,d)(A,C) \tkzGetPoint{Q}
				\tkzDrawSegments(B,P D,Q)
				\tkzLabelPoint[small label,right](P){P}
				\tkzLabelPoint[small label,below left](Q){Q}
				\tkzMarkAngle[orange,mark=|,size=0.5](A,D,Q)
				\tkzMarkAngle[orange,mark=|,size=0.6](Q,D,C)
				\tkzMarkAngle[cyan,mark=||,size=0.5](P,B,A)
				\tkzMarkAngle[cyan,mark=||,size=0.6](C,B,P)
			\end{tikzpicture}
		\end{figure}
		\begin{itemize}
			\item $AE=4,EQ=2,QP=3$
			\item Find $PC$
		\end{itemize}
		\column{0.5\textwidth}
		\begin{enumerate}
			\pause
			\item $\frac{AE}{EC} = \frac{[ABD]}{[CBD]} = \frac{\frac{1}{2}AB\cdot AD \sin \angle BAD}{\frac{1}{2}BC \cdot CD \sin \angle BCD} = \frac{AP}{PC}\cdot \frac{AQ}{QC}$
				\pause
			\item Let $PC = x$, we have  $\frac{4}{5+x} = \frac{9}{x} \cdot \frac{6}{3+x}$
				\pause
			\item Solving gives $x = -\frac{9}{2}$ or $x = 15$
				\pause
			\item Hence $PC = 15$
		\end{enumerate}
	\end{columns}
\end{frame}

\begin{frame}{Extended Angle Bisector Theorem}
	We have the following extended version of the Angle Bisector Theorem
	\begin{theorem}[Extended Angle Bisector Theorem]
		Let $BD$ be the external angle bisector of $\angle ABC$, then $AB:BC = AD:DC$.
	\end{theorem}
	\begin{figure}
		\centering
		\begin{tikzpicture}
			\def\r{1.6}
			\tkzDefPoint(0,0){O}
			\tkzDefPoint(180:\r){A}
			\tkzDefPoint(130:\r){B}
			\tkzDefPoint(0:\r){C}
			\coordinate (E) at ($(C)!1.4!(B)$);
			\tkzDefLine[bisector](E,B,A) \tkzGetPoint{b}
			\tkzInterLL(B,b)(A,C) \tkzGetPoint{D}
			\tkzLabelPoint[below](A){A}
			\tkzLabelPoint[right](C){C}
			\tkzLabelPoint[above](B){B}
			\tkzLabelPoint[left](D){D}
			\tkzDrawSegment(A,B)
			\tkzDrawSegment(E,C)
			\tkzDrawSegment(C,D)
			\tkzDrawSegment(B,D)
			\tkzMarkAngle[size=0.6,cyan,mark=|](D,B,A)
			\tkzMarkAngle[size=0.5,cyan,mark=|](E,B,D)
		\end{tikzpicture}
	\end{figure}
	The proof is left as an exercise. :)
\end{frame}

\begin{frame}{Prelim 2022 Q19}
	\begin{block}{Prelim 2022 Q19 (Modified)}
		In $\triangle A B C, A B<A C$. The internal bisector of $\angle B A C$ meets $B C$ at $D$, while the external bisector of $\angle B A C$ meets $C B$ produced at $E$. If $EB=10$ and $BD=5$, find the length of $DC$.
	\end{block}
	\pause
	\begin{figure}
		\centering
		\begin{tikzpicture}
			\def\r{1.6}
			\tkzDefPoint(0,0){O}
			\tkzDefPoint(180:\r){B}
			\tkzDefPoint(130:\r){A}
			\tkzDefPoint(0:\r){C}
			\coordinate (F) at ($(C)!1.4!(A)$);
			\tkzDefLine[bisector](F,A,B) \tkzGetPoint{a}
			\tkzDefLine[bisector](B,A,C) \tkzGetPoint{a'}
			\tkzInterLL(A,a)(B,C) \tkzGetPoint{E}
			\tkzInterLL(A,a')(B,C) \tkzGetPoint{D}
			\tkzLabelPoint[below](B){B}
			\tkzLabelPoint[right](C){C}
			\tkzLabelPoint[above](A){A}
			\tkzLabelPoint[left](E){E}
			\tkzLabelPoint[below](D){D}
			\tkzDrawSegment(A,B)
			\tkzDrawSegment(F,C)
			\tkzDrawSegment(C,E)
			\tkzDrawSegment(B,E)
			\tkzDrawSegment(A,E)
			\tkzDrawSegment(A,D)
			\tkzMarkAngle[size=0.5,orange,mark=|](F,A,E)
			\tkzMarkAngle[size=0.6,orange,mark=|](E,A,B)
			\tkzMarkAngle[size=0.5,cyan,mark=||](B,A,D)
			\tkzMarkAngle[size=0.6,cyan,mark=||](D,A,C)
		\end{tikzpicture}
	\end{figure}
\end{frame}

\begin{frame}{Prelim 2022 Q19 (Solution)}
	\begin{columns}
		\column{0.5\textwidth}
		\begin{figure}
			\centering
			\begin{tikzpicture}
				\def\r{1.3}
				\tkzDefPoint(0,0){O}
				\tkzDefPoint(180:\r){B}
				\tkzDefPoint(130:\r){A}
				\tkzDefPoint(0:\r){C}
				\coordinate (F) at ($(C)!1.4!(A)$);
				\tkzDefLine[bisector](F,A,B) \tkzGetPoint{a}
				\tkzDefLine[bisector](B,A,C) \tkzGetPoint{a'}
				\tkzInterLL(A,a)(B,C) \tkzGetPoint{E}
				\tkzInterLL(A,a')(B,C) \tkzGetPoint{D}
				\tkzLabelPoint[below](B){B}
				\tkzLabelPoint[right](C){C}
				\tkzLabelPoint[above](A){A}
				\tkzLabelPoint[left](E){E}
				\tkzLabelPoint[below](D){D}
				\tkzDrawSegment(A,B)
				\tkzDrawSegment(F,C)
				\tkzDrawSegment(C,E)
				\tkzDrawSegment(B,E)
				\tkzDrawSegment(A,E)
				\tkzDrawSegment(A,D)
				\tkzMarkAngle[size=0.5,orange,mark=|](F,A,E)
				\tkzMarkAngle[size=0.6,orange,mark=|](E,A,B)
				\tkzMarkAngle[size=0.5,cyan,mark=||](B,A,D)
				\tkzMarkAngle[size=0.6,cyan,mark=||](D,A,C)
			\end{tikzpicture}
			\begin{itemize}
				\item $EB = 10$, $BD = 5$
				\item Find $DC$
			\end{itemize}
		\end{figure}
		\column{0.5\textwidth}
		\pause
		\begin{enumerate}
			\item Let $DC=x$
				\pause
			\item $\frac{EB}{EC} = \frac{AB}{BC} = \frac{BD}{DC}$ 
				\pause
			\item $\frac{10}{15+x} = \frac{5}{x}$
				\pause
			\item Solving yields $x = 15$.
		\end{enumerate}
	\end{columns}	
\end{frame}

\section{Cosine's Law}
\begin{frame}{Cosine's Law}
	\begin{theorem}[Cosine's Law]
		In $\triangle ABC$,  $c^2 = a^2 + b^2 - 2ab\cos C$. Equivalently, $\cos C = \frac{a^2+b^2-c^2}{2ab}$
	\end{theorem}
	\pause
	Proof:
	\begin{figure}
		\centering
		\begin{tikzpicture}
			\tkzDefPoint(-1,0){A}		
			\tkzDefPoint(2,0){B}		
			\tkzDefPoint(0,1){C}		
			\tkzDrawPolygon(A,B,C)
			\tkzDefPointBy[projection=onto A--B](C) \tkzGetPoint{H}
			\tkzLabelPoint[left](A){C}
			\tkzLabelPoint[right](B){B}
			\tkzLabelPoint[above](C){A}
			\tkzLabelPoint[below](H){H}
			\tkzDrawSegment(C,H)
		\end{tikzpicture}
	\end{figure}
	Observe that $c^2=AH^2+HB^2=b\sin C^2+(a-b\cos C)^2=a^2+b^2-2ab\cos C$
	\pause
	\begin{block}{Exercise}
		Prove the Stewart's Theorem with the Cosine's Law
	\end{block}
\end{frame}

\section{Ceva's Theorem}
\begin{frame}{Ceva's Theorem}
	In a triangle, a \emph{cevian} is a line joining a vertex of a triangle to a point on the interior of the opposite side. A natural question is when three cevians of a triangle concurs. This is answered by Ceva's theorem.
	\begin{theorem}[Ceva's Theorem]
		Let $\overline{AX}$,  $\overline{BY}$,  $\overline{CZ}$ be cevians of a triangle $ABC$. They concur if and only if
		\begin{equation*}
			\frac{BX}{XC} \cdot \frac{CY}{YA} \cdot \frac{AZ}{ZB} = 1
		\end{equation*}
	\end{theorem}
	\begin{figure}
		\centering
		\begin{tikzpicture}
			\tkzDefPoint(0,0){A}
			\tkzDefPoint(-2,-2.5){B}
			\tkzDefPoint(3,-2.5){C}
			\coordinate (X) at ($(B)!0.333333333333!(C)$);
			\coordinate (Z) at ($(B)!0.333333333333!(A)$);
			\coordinate (Y) at ($(C)!0.5!(A)$);
			\tkzDrawPolygon(A,B,C)
			\tkzDrawSegment(A,X)
			\tkzDrawSegment(B,Y)
			\tkzDrawSegment(C,Z)
			\tkzLabelPoints[above](A){A}
			\tkzLabelPoints[left](B){B}
			\tkzLabelPoints[right](C){C}
			\tkzLabelPoints[below](X){X}
			\tkzLabelPoints[right](Y){Y}
			\tkzLabelPoints[left](Z){Z}
		\end{tikzpicture}
	\end{figure}
\end{frame}

\begin{frame}{Ceva's Theorem (proof)}
	\begin{columns}
		\column{0.5\textwidth}
		\begin{figure}
			\centering
			\begin{tikzpicture}[scale=0.8]
				\tkzDefPoint(0,0){A}
				\tkzDefPoint(-2,-2.5){B}
				\tkzDefPoint(3,-2.5){C}
				\coordinate (X) at ($(B)!0.333333333333!(C)$);
				\coordinate (Z) at ($(B)!0.333333333333!(A)$);
				\coordinate (Y) at ($(C)!0.5!(A)$);
				\tkzDrawPolygon(A,B,C)
				\tkzDrawSegment(A,X)
				\tkzDrawSegment(B,Y)
				\tkzDrawSegment(C,Z)
				\tkzLabelPoints[above](A){A}
				\tkzLabelPoints[left](B){B}
				\tkzLabelPoints[right](C){C}
				\tkzLabelPoints[below](X){X}
				\tkzLabelPoints[right](Y){Y}
				\tkzLabelPoints[left](Z){Z}
				\tkzInterLL(A,X)(B,Y) \tkzGetPoint{P}
				\tkzLabelPoint[above right](P){P}
			\end{tikzpicture}
		\end{figure}
		\column{0.5\textwidth}
		I will only prove the forward direction, i.e. if three cevians concur, then the identity $\frac{BX}{XC} \cdot \frac{CY}{YA} \cdot \frac{AZ}{ZB} = 1$ \\
		Proof:
		\pause
		\begin{enumerate}
			\item $\frac{[ABX]}{[AXC]} = \frac{BX}{XC}$ and $\frac{[BPX]}{[CPX]} = \frac{BX}{XC}$.
				\pause
			\item Hence $\frac{[APB]}{[APC]} = \frac{BX}{XC}$
				\pause
			\item Similarly $\frac{[BPA]}{[BPC]} = \frac{AY}{YC}$ and  $\frac{[CPB]}{[CPA]} = \frac{ZB}{ZA}$.
				\pause
			\item Multiplying the above three equations gives $\frac{BX}{XC} \cdot \frac{CY}{YA} \cdot \frac{AZ}{ZB} = 1$
		\end{enumerate}
	\end{columns}
\end{frame}

\begin{frame}{Trigonometric Form of Ceva's Theorem}
	\begin{block}{Trigonometric Form of Ceva's Theorem}
		Let $\overline{AX}$,  $\overline{BY}$,  $\overline{CZ}$ be cevians of a triangle  $ABC$. They concur if and only if
		\begin{equation*}
			\frac{\sin \angle BAX \sin \angle CBY \sin \angle ACZ}{\sin \angle XAC \sin \angle YBA \sin \angle ZCB } = 1
		\end{equation*}
	\end{block}
	\begin{figure}
		\centering
		\begin{tikzpicture}[scale=0.8]
			\tkzDefPoint(0,0){A}
			\tkzDefPoint(-2,-2.5){B}
			\tkzDefPoint(3,-2.5){C}
			\coordinate (X) at ($(B)!0.333333333333!(C)$);
			\coordinate (Z) at ($(B)!0.333333333333!(A)$);
			\coordinate (Y) at ($(C)!0.5!(A)$);
			\tkzDrawPolygon(A,B,C)
			\tkzDrawSegment(A,X)
			\tkzDrawSegment(B,Y)
			\tkzDrawSegment(C,Z)
			\tkzLabelPoints[above](A){A}
			\tkzLabelPoints[left](B){B}
			\tkzLabelPoints[right](C){C}
			\tkzLabelPoints[below](X){X}
			\tkzLabelPoints[right](Y){Y}
			\tkzLabelPoints[left](Z){Z}
			\tkzInterLL(A,X)(B,Y) \tkzGetPoint{P}
			\tkzLabelPoint[above right](P){P}
		\end{tikzpicture}
	\end{figure}
	The proof is a direct application of the law of Sine and is left as an exercise.
\end{frame}

\begin{frame}{Existence of orthocenter, incenter, centroid}
	Have you ever wondered why the three altitudes, angle bisectors, or medians must concur at the same point? \\
	\pause
	For the orthocentre (of acute triangle), we check
	\begin{equation*}
		\frac{\sin(90^o-B)\sin(90^o-C)\sin(90^o-A)}{\sin(90^o-C)\sin(90^o-A)\sin(90^o-B)} = 1
	\end{equation*}
	\pause
	For the incenter, we check 
	\begin{equation*}
		\frac{\sin \frac{1}{2}A \sin \frac{1}{2}B \sin \frac{1}{2}C}{\sin \frac{1}{2}A \sin \frac{1}{2}B \sin \frac{1}{2}C} = 1
	\end{equation*}
	\pause
	For the centroid, we have
	\begin{equation*}
		\frac{1}{1} \frac{1}{1} \frac{1}{1} = 1
	\end{equation*}
	We no longer have to take the existence of our centers for granted!
	\pause
	\begin{block}{Where is our circumcentre?}
		Why don't we prove the circumcentre case as well?
	\end{block}
\end{frame}

\section{Menelaus's Theorem}
\begin{frame}{Menelaus's Theorem}
	\begin{theorem}[Menelaus's Theorem]
		Let $X$, $Y$, $Z$ be points on lines $BC$, $CA$, $AB$ in a triangle $ABC$, distinct from its vertices. Then $X$, $Y$, $Z$ are collinear if and only if
		\begin{equation*}
			\frac{BX}{XC} \cdot \frac{CY}{YA} \cdot \frac{AZ}{ZB} = -1
		\end{equation*}
		where lengths are directed.
	\end{theorem}
	\begin{figure}
		\centering
		\begin{tikzpicture}[scale=0.7,small label]
			\tkzDefPoint(0,0){A}
			\tkzDefPoint(-2,-2.5){B}
			\tkzDefPoint(3,-2.5){C}
			\coordinate (Z) at ($(B)!0.333333333333!(A)$);
			\coordinate (Y) at ($(C)!0.7!(A)$);
			\tkzInterLL(Z,Y)(B,C) \tkzGetPoint{X}
			\tkzDrawPolygon(A,B,C)
			\tkzLabelPoints[small label,above](A){A}
			\tkzLabelPoints[small label,below](B){B}
			\tkzLabelPoints[small label,right](C){C}
			\tkzLabelPoints[small label,left](X){X}
			\tkzLabelPoints[small label,right](Y){Y}
			\tkzLabelPoints[small label,above left](Z){Z}
			\tkzDrawSegment[very thick](X,Y)
			\tkzDrawSegment[dashed](B,X)
			\tkzDrawPoints(X,Z,Y)
		\end{tikzpicture}
		\begin{tikzpicture}[scale=0.5,small label]
			\tkzDefPoint(1,0){A}
			\tkzDefPoint(-0.5,-2.5){B}
			\tkzDefPoint(3,-2.5){C}
			\coordinate (Z) at ($(B)!2!(A)$);
			\tkzInterLL(X,Z)(A,C) \tkzGetPoint{Y}
			\tkzInterLL(Z,Y)(B,C) \tkzGetPoint{X}
			\tkzDrawPolygon(A,B,C)
			\tkzLabelPoints[small label,above](A){A}
			\tkzLabelPoints[small label,below](B){B}
			\tkzLabelPoints[small label,right](C){C}
			\tkzLabelPoints[small label,left](X){X}
			\tkzLabelPoints[small label,left](Y){Y}
			\tkzLabelPoints[small label,above left](Z){Z}
			\tkzDrawSegment[very thick](X,Z)
			\tkzDrawSegment[dashed](B,X)
			\tkzDrawSegment[dashed](A,Y)
			\tkzDrawSegment[dashed](A,Z)
			\tkzDrawPoints(X,Z,Y)
		\end{tikzpicture}
	\end{figure}
\end{frame}

\begin{frame}{Proof of Menelaus's Theorem}
	\begin{columns}
		\column{0.5\textwidth}	
		\begin{figure}
			\centering
			\begin{tikzpicture}[scale=0.75,small label]
				\tkzDefPoint(0,0){A}
				\tkzDefPoint(-2,-2.5){B}
				\tkzDefPoint(3,-2.5){C}
				\coordinate (Z) at ($(B)!0.333333333333!(A)$);
				\coordinate (Y) at ($(C)!0.7!(A)$);
				\tkzInterLL(Z,Y)(B,C) \tkzGetPoint{X}
				\tkzDrawPolygon(A,B,C)
				\tkzLabelPoints[small label,above](A){A}
				\tkzLabelPoints[small label,below](B){B}
				\tkzLabelPoints[small label,right](C){C}
				\tkzLabelPoints[small label,left](X){X}
				\tkzLabelPoints[small label,right](Y){Y}
				\tkzLabelPoints[small label,above left](Z){Z}
				\tkzDrawSegment[very thick](X,Y)
				\tkzDrawSegment[dashed](B,X)
				\tkzDrawPoints(X,Z,Y)
				\tkzDefPointBy[projection=onto X--Y](A) \tkzGetPoint{A'}
				\tkzDefPointBy[projection=onto X--Y](B) \tkzGetPoint{B'}
				\tkzDefPointBy[projection=onto X--Y](C) \tkzGetPoint{C'}
				\onslide<2> {
					\tkzDrawSegments[new,dashed](A,A' B,B' C,C')
					\tkzDrawSegment[new,dashed](Y,C')
					\tkzLabelPoints[new,small label](A')
					\tkzLabelPoints[new,small label,above](B')
					\tkzLabelPoints[new,small label,above](C')
					\tkzDrawPoints[new](B',A',C')
				}
				\onslide<3-> {
					\tkzDrawSegments[dashed](A,A' B,B' C,C')
					\tkzDrawSegment[dashed](Y,C')
					\tkzLabelPoints[small label](A')
					\tkzLabelPoints[small label,above](B')
					\tkzLabelPoints[small label,above](C')
					\tkzDrawPoints(B',A',C')
				}
			\end{tikzpicture}
		\end{figure}
		\column{0.5\textwidth}
		Proof of the first case:
		\pause
		\begin{enumerate}
			\item Drop a perpendicular line from $A$ to $A'$, $B$ to $B'$, $C$ to $C'$ on $XY$.
				\pause
			\item We have $\frac{CY}{YA} = \frac{CC'}{AA'}$, $\frac{AA'}{BB'} =  \frac{AZ}{ZB}$, and $\frac{BX}{XC} = -\frac{BB'}{CC'}$
				\pause
			\item Multiplying all of them gives the Menelaus's Theorem.
		\end{enumerate}
		\pause
		The proof of the second case is identical.
	\end{columns}	
\end{frame}

\section{The Centroid Triangle}
\begin{frame}{The Centroid Triangle}
	This slide serves to, yet again, stress the importance of the area ratios.	
	\begin{theorem}[The Centroid Division]
		The medians divides the triangle into $6$ equal parts.
	\end{theorem}
	\begin{figure}
		\centering
		\begin{tikzpicture}[small label,scale=1.1]
			\tkzDefPoint(0.5,0.5){A}
			\tkzDefPoint(-2,-1.5){B}
			\tkzDefPoint(1,-1.5){C}
			\tkzDrawPolygon(A,B,C)
			\tkzDefCentroid(A,B,C) \tkzGetPoint{G}
			\tkzInterLL(A,G)(B,C) \tkzGetPoint{M}
			\tkzInterLL(B,G)(A,C) \tkzGetPoint{N}
			\tkzInterLL(C,G)(A,B) \tkzGetPoint{L}
			\tkzDrawSegment(A,M)
			\tkzDrawSegment(B,N)
			\tkzDrawSegment(C,L)
			\tkzLabelPoint[small label,above](G){G}
			\tkzLabelPoint[small label,left](B){B}
			\tkzLabelPoint[small label,right](C){C}
			\tkzLabelPoint[small label,above](A){A}
			\tkzLabelPoint[small label,below](M){M}
			\tkzLabelPoint[small label,right](N){N}
			\tkzLabelPoint[small label,left](L){L}
		\end{tikzpicture}
	\end{figure}
	\begin{block}{Exercise}
		Prove the claim. Hence, show that $AG=2GM$,  $CG=2LG$,  $BG=2GN$
	\end{block}
\end{frame}

\section{Practice Problems}
\begin{frame}{Practice Problems}
	\begin{block}{Question}
		Point $P$ is on side $AB$ of right angled $\triangle ABC$ with $B$ as the right angled. Point $Q$ is on $AC$ such that $PQ$ is perpendicular to $AC$. It is given that $BC=3$ and $BP=PA=2$. Find the length $BQ$.
	\end{block}	
	\begin{block}{Prelim 2020 Q16}
		$\triangle ABC$ is right-angled at $B$, with $AB=1$ and $BC=3$. $E$ is the foot of perpendicular from $B$ to $AC$. $BA$ and $BE$ are produced to $D$ and $F$ respectively such that $D$,$F$,$C$ are collinear and $\angle DAF = \angle BAC$. Find the length of $AD$.
	\end{block}
	\begin{block}{Prelim 2019 Q10}
		In $\triangle ABC$, $AB < AC$. Let $H$ be the orthocentre of $\triangle ABC$, and  $D$ be the foot of the perpendicular from $A$ to $BC$. If $AH=4$,$HD=3$ and $BC=12$, find the length of $BD$.
	\end{block}
\end{frame}

\begin{frame}{The End}
	\centering \Large
	\emph{Thank You!}
\end{frame}

\end{document}
